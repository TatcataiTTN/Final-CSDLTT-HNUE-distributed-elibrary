% ============================================================================
% KẾT LUẬN VÀ PHƯƠNG HƯỚNG PHÁT TRIỂN
% ============================================================================
\chapter{KẾT LUẬN VÀ PHƯƠNG HƯỚNG PHÁT TRIỂN}
\chaptermark{KẾT LUẬN}

\section{Kết luận}

Qua quá trình nghiên cứu và thực hiện đề tài ``Xây dựng hệ thống E-Library Phân tán nhiều cơ sở'', nhóm đã hoàn thành các mục tiêu đề ra và đạt được những kết quả quan trọng.

\subsection{Những kết quả đạt được}

\subsubsection{Về mặt hệ thống}

Nhóm đã xây dựng thành công một hệ thống thư viện điện tử phân tán với kiến trúc 4 node, bao gồm một Central Hub (Trung tâm điều phối) đặt tại thư mục Nhasach và ba chi nhánh vùng tại Hà Nội, Đà Nẵng và Thành phố Hồ Chí Minh. Mỗi node hoạt động độc lập với cơ sở dữ liệu riêng biệt, đồng thời có khả năng đồng bộ dữ liệu với Central Hub thông qua các API REST được thiết kế đặc biệt cho mục đích này.

Hệ thống cơ sở dữ liệu MongoDB được triển khai theo mô hình Replica Set với Docker, bao gồm 3 instance chạy trên các cổng 27017, 27018 và 27019. Cấu hình này đảm bảo tính sẵn sàng cao nhờ cơ chế tự động chuyển đổi dự phòng (automatic failover) khi node Primary gặp sự cố. Các thông số Read Preference và Write Concern được tối ưu để cân bằng giữa hiệu năng đọc và tính nhất quán dữ liệu.

Về quy mô dữ liệu, hệ thống hiện đang quản lý:
\begin{itemize}
    \item \textbf{Tổng số sách:} 1.018 đầu sách (Central: 509, Hà Nội: 200, Đà Nẵng: 163, TP.HCM: 146)
    \item \textbf{Người dùng:} 42 tài khoản đã đăng ký với đầy đủ thông tin cá nhân
    \item \textbf{Đơn mượn:} 111 đơn mượn sách đã được xử lý với các trạng thái khác nhau
    \item \textbf{Phân loại:} Sách được phân thành nhiều thể loại như Văn học, Khoa học, Lịch sử, Kinh tế, v.v.
\end{itemize}

\subsubsection{Về mặt chức năng nghiệp vụ}

Hệ thống quản lý sách được xây dựng hoàn chỉnh với đầy đủ các thao tác Tạo-Đọc-Sửa-Xóa (CRUD), kèm theo cơ chế kiểm tra dữ liệu đầu vào (validation) để đảm bảo tính toàn vẹn. Chức năng tìm kiếm toàn văn (full-text search) sử dụng TEXT index của MongoDB, hỗ trợ tìm kiếm tiếng Việt có dấu. Sách được phân loại theo thể loại và chi nhánh, với thông tin chi tiết về số lượng tồn kho và giá thuê theo ngày.

Hệ thống người dùng triển khai mô hình phân quyền theo vai trò (RBAC) với hai nhóm: quản trị viên có quyền quản lý toàn bộ hệ thống, và khách hàng chỉ có quyền mượn sách và xem thông tin cá nhân. Xác thực người dùng sử dụng chuẩn JWT (JSON Web Token) với thời hạn 24 giờ, mật khẩu được mã hóa bằng thuật toán bcrypt với độ phức tạp 12, đảm bảo an toàn trước các cuộc tấn công brute-force. Mỗi người dùng có một tài khoản số dư để thanh toán phí mượn sách.

Quy trình mượn sách được thiết kế trực quan với giỏ hàng lưu trữ trong session, cho phép khách hàng chọn nhiều sách trước khi thanh toán. Khi xác nhận mượn, hệ thống tự động kiểm tra và trừ số dư tài khoản, tạo đơn mượn với thông tin chi tiết về ngày mượn, ngày trả dự kiến và tổng tiền. Lịch sử mượn sách được lưu trữ đầy đủ để khách hàng theo dõi và quản trị viên thống kê.

Trang Dashboard dành cho quản trị viên hiển thị các biểu đồ thống kê trực quan bằng thư viện Chart.js, bao gồm: tổng quan về số sách, người dùng, đơn mượn và doanh thu; biểu đồ tròn thể hiện phân bố sách theo thể loại; biểu đồ cột so sánh số lượng sách giữa các chi nhánh; và biểu đồ đường thể hiện xu hướng mượn sách theo từng tháng.

\subsubsection{Về mặt kỹ thuật NoSQL}

\begin{enumerate}
    \item \textbf{Aggregation Pipeline - 8 endpoints thống kê}:
    \begin{itemize}
        \item Sử dụng 10+ operators: \$match, \$group, \$sort, \$lookup, \$unwind, \$project, \$addFields, \$facet, \$bucket, \$limit
        \item \$lookup JOIN giữa orders và users collection
        \item \$facet cho multiple aggregations trong một query
        \item \$bucket cho phân nhóm theo khoảng giá
    \end{itemize}

    \item \textbf{Map-Reduce - 5 operations phân tích}:
    \begin{itemize}
        \item borrow\_stats: Thống kê mượn sách theo user
        \item revenue\_by\_user: Doanh thu theo người dùng
        \item books\_by\_category: Phân bố sách theo thể loại
        \item daily\_activity: Hoạt động theo ngày
        \item location\_performance: Hiệu suất theo chi nhánh
    \end{itemize}

    \item \textbf{Indexing Strategy - 7 indexes tối ưu}:
    \begin{itemize}
        \item Unique indexes: username, bookCode
        \item Compound index: location + bookName
        \item Single field: bookGroup, location, borrowCount
        \item TEXT index: bookName + author + publisher
    \end{itemize}
\end{enumerate}

\subsubsection{Về mặt hiệu năng}

Nhóm đã thực hiện đo lường hiệu năng (benchmark) với 50 lần lặp cho mỗi kịch bản truy vấn, sử dụng dữ liệu thực của hệ thống trên MongoDB phiên bản 8.0.16. Kết quả cho thấy hệ thống đáp ứng tốt các yêu cầu về tốc độ phản hồi cho ứng dụng web tương tác:

\begin{table}[H]
\centering
\caption{Tổng hợp kết quả đo lường hiệu năng (benchmark)}
\begin{tabular}{|l|r|r|}
\hline
\textbf{Chỉ số} & \textbf{Giá trị} & \textbf{Đánh giá} \\
\hline
Truy vấn nhanh nhất & 0.880 ms & Xuất sắc \\
Truy vấn chậm nhất & 6.460 ms & Chấp nhận được \\
Trung bình 10 loại truy vấn & 3.430 ms & Tốt \\
Throughput cao nhất & 1,136 thao tác/giây & Phù hợp quy mô nhỏ-vừa \\
\hline
\end{tabular}
\end{table}

Truy vấn nhanh nhất là Range Query kết hợp với Sort (0.880ms) nhờ tận dụng tốt index trên trường giá thuê. Truy vấn chậm nhất là Single Location Query (6.460ms) do phải trả về nhiều documents. Mức throughput 1,136 thao tác/giây hoàn toàn đủ cho một hệ thống thư viện quy mô vừa với vài trăm người dùng đồng thời.

\subsection{Những điểm còn hạn chế}

\begin{enumerate}
    \item \textbf{Dataset thử nghiệm còn nhỏ}:
    \begin{itemize}
        \item 909 sách chưa đủ để stress test sharding performance
        \item Cần dataset 100K+ records để đánh giá chunk migration
        \item Benchmark chưa mô phỏng concurrent users
    \end{itemize}

    \item \textbf{Chưa triển khai TLS/SSL encryption}:
    \begin{itemize}
        \item Kết nối MongoDB chưa được mã hóa
        \item JWT token truyền qua HTTP (chưa HTTPS)
        \item Cần bổ sung cho production deployment
    \end{itemize}

    \item \textbf{Đồng bộ dữ liệu thủ công}:
    \begin{itemize}
        \item Sync giữa Central và branches cần admin trigger
        \item Chưa có real-time synchronization
        \item Conflict resolution chưa hoàn chỉnh
    \end{itemize}

    \item \textbf{Chưa có cơ chế backup tự động}:
    \begin{itemize}
        \item Backup thủ công với mongodump
        \item Chưa có scheduled backup
        \item Thiếu point-in-time recovery
    \end{itemize}
\end{enumerate}

\subsection{Kiến thức và kỹ năng đạt được}

\begin{enumerate}
    \item \textbf{Hiểu sâu về định lý CAP}:
    \begin{itemize}
        \item Consistency, Availability, Partition Tolerance trade-offs
        \item MongoDB là CP system với tunable consistency
        \item Read/Write Concern configuration
    \end{itemize}

    \item \textbf{Thành thạo MongoDB operations}:
    \begin{itemize}
        \item CRUD với PHP MongoDB Library
        \item Aggregation Pipeline optimization
        \item Index design và query analysis
    \end{itemize}

    \item \textbf{Triển khai containerized applications}:
    \begin{itemize}
        \item Docker Compose multi-container orchestration
        \item Volume persistence và networking
        \item Health checks và restart policies
    \end{itemize}

    \item \textbf{Bảo mật web applications}:
    \begin{itemize}
        \item JWT token-based authentication
        \item Password hashing best practices
        \item Role-based access control implementation
    \end{itemize}
\end{enumerate}

\section{Phương hướng phát triển}

\subsection{Cải tiến ngắn hạn}

\begin{enumerate}
    \item \textbf{Nâng cấp bảo mật}:
    \begin{itemize}
        \item Triển khai TLS/SSL cho MongoDB connections
        \item HTTPS cho tất cả endpoints
        \item API rate limiting chống DDoS
        \item Two-Factor Authentication (2FA)
    \end{itemize}

    \item \textbf{Real-time synchronization}:
    \begin{itemize}
        \item MongoDB Change Streams cho real-time sync
        \item WebSocket notifications cho UI updates
        \item Conflict resolution với vector clocks
    \end{itemize}

    \item \textbf{Automated operations}:
    \begin{itemize}
        \item Scheduled backup với mongodump
        \item Point-in-time recovery setup
        \item Monitoring với Prometheus/Grafana
        \item Alerting cho system health
    \end{itemize}
\end{enumerate}

\subsection{Phát triển trung hạn}

\begin{enumerate}
    \item \textbf{Tích hợp Redis Cache}:
    \begin{itemize}
        \item Cache layer cho frequently accessed data
        \item Session storage với Redis
        \item Giảm tải 70-80\% read operations từ MongoDB
        \item Cache invalidation strategy
    \end{itemize}

    \item \textbf{Microservices architecture}:
    \begin{itemize}
        \item Tách thành các services độc lập
        \item API Gateway với Kong/Traefik
        \item Service discovery với Consul
        \item Message queue với RabbitMQ
    \end{itemize}

    \item \textbf{Triển khai Sharding thực sự}:
    \begin{itemize}
        \item Compound Shard Key: \{location: 1, bookCode: 1\}
        \item Config servers và Mongos routers
        \item Zone-based sharding cho data locality
        \item Chunk balancing optimization
    \end{itemize}
\end{enumerate}

\subsection{Phát triển dài hạn}

\begin{enumerate}
    \item \textbf{Cloud deployment}:
    \begin{itemize}
        \item MongoDB Atlas cho managed cluster
        \item AWS/GCP/Azure auto-scaling
        \item CDN cho static assets (Cloudflare)
        \item Load balancer với Nginx/HAProxy
    \end{itemize}

    \item \textbf{Mobile applications}:
    \begin{itemize}
        \item iOS/Android app với React Native
        \item Push notifications cho due date reminders
        \item QR code scanning cho quick book lookup
        \item Offline mode với local SQLite
    \end{itemize}

    \item \textbf{Tích hợp AI/ML}:
    \begin{itemize}
        \item Recommendation engine cho book suggestions
        \item Demand forecasting để optimize inventory
        \item Chatbot hỗ trợ tra cứu sách (OpenAI/Claude)
        \item OCR cho digitization của sách giấy
    \end{itemize}

    \item \textbf{Analytics platform}:
    \begin{itemize}
        \item Data warehouse với MongoDB Analytics
        \item Business Intelligence dashboards
        \item User behavior analysis
        \item A/B testing infrastructure
    \end{itemize}
\end{enumerate}

% ============================================================================
% TÀI LIỆU THAM KHẢO
% ============================================================================
\chapter*{TÀI LIỆU THAM KHẢO}
\addcontentsline{toc}{chapter}{TÀI LIỆU THAM KHẢO}

\section*{Tài liệu MongoDB}

\begin{enumerate}[label={[\arabic*]}]
    \item MongoDB Inc. (2025). \textit{MongoDB Manual - Sharding}. \url{https://www.mongodb.com/docs/manual/sharding/}

    \item MongoDB Inc. (2025). \textit{MongoDB Manual - Replication}. \url{https://www.mongodb.com/docs/manual/replication/}

    \item MongoDB Inc. (2025). \textit{MongoDB Manual - Aggregation Pipeline}. \url{https://www.mongodb.com/docs/manual/aggregation/}

    \item MongoDB Inc. (2025). \textit{MongoDB Manual - Map-Reduce}. \url{https://www.mongodb.com/docs/manual/core/map-reduce/}

    \item MongoDB Inc. (2025). \textit{MongoDB Manual - Indexes}. \url{https://www.mongodb.com/docs/manual/indexes/}
\end{enumerate}

\section*{Tài liệu PHP và Development}

\begin{enumerate}[label={[\arabic*]}, resume]
    \item The PHP Group. (2025). \textit{PHP Manual - MongoDB Driver}. \url{https://www.php.net/manual/en/set.mongodb.php}

    \item MongoDB Inc. (2025). \textit{mongodb/mongodb PHP Library Documentation}. \url{https://www.mongodb.com/docs/php-library/current/}

    \item Docker Inc. (2025). \textit{Docker Compose Documentation}. \url{https://docs.docker.com/compose/}

    \item Firebase. (2025). \textit{PHP-JWT Library}. \url{https://github.com/firebase/php-jwt}

    \item Chart.js Contributors. (2025). \textit{Chart.js Documentation v4.4}. \url{https://www.chartjs.org/docs/latest/}
\end{enumerate}

\section*{Sách tham khảo}

\begin{enumerate}[label={[\arabic*]}, resume]
    \item Bradshaw, S., Brazil, E., \& Chodorow, K. (2019). \textit{MongoDB: The Definitive Guide} (3rd ed.). O'Reilly Media. ISBN: 978-1-4919-5446-1.

    \item Kleppmann, M. (2017). \textit{Designing Data-Intensive Applications}. O'Reilly Media. ISBN: 978-1-4493-7332-0.

    \item Sadalage, P. J., \& Fowler, M. (2012). \textit{NoSQL Distilled: A Brief Guide to the Emerging World of Polyglot Persistence}. Addison-Wesley. ISBN: 978-0-321-82662-6.

    \item Tanenbaum, A. S., \& Van Steen, M. (2017). \textit{Distributed Systems: Principles and Paradigms} (3rd ed.). Pearson. ISBN: 978-1-5309-4117-3.
\end{enumerate}

\section*{Bài báo khoa học}

\begin{enumerate}[label={[\arabic*]}, resume]
    \item Gilbert, S., \& Lynch, N. (2002). Brewer's Conjecture and the Feasibility of Consistent, Available, Partition-Tolerant Web Services. \textit{ACM SIGACT News}, 33(2), 51-59.

    \item Ongaro, D., \& Ousterhout, J. (2014). In Search of an Understandable Consensus Algorithm. \textit{Proceedings of the 2014 USENIX ATC}, 305-319.

    \item DeCandia, G., et al. (2007). Dynamo: Amazon's Highly Available Key-value Store. \textit{Proceedings of SOSP'07}, 205-220.
\end{enumerate}

\section*{Tiêu chuẩn và RFC}

\begin{enumerate}[label={[\arabic*]}, resume]
    \item Jones, M., Bradley, J., \& Sakimura, N. (2015). \textit{RFC 7519 - JSON Web Token (JWT)}. IETF. \url{https://tools.ietf.org/html/rfc7519}
\end{enumerate}

\section*{Giáo trình}

\begin{enumerate}[label={[\arabic*]}, resume]
    \item Nguyễn Duy Hải. (2025). \textit{Bài giảng Cơ sở dữ liệu tiên tiến - NoSQL \& Distributed Systems}. Trường Đại học Sư phạm Hà Nội.
\end{enumerate}

% ============================================================================
% PHỤ LỤC
% ============================================================================
\appendix
\chapter{PHỤ LỤC}

\section{Bảng ký hiệu và chữ viết tắt}

\begin{longtable}{|c|l|p{8cm}|}
\caption{Bảng từ viết tắt và ký hiệu} \\
\hline
\textbf{STT} & \textbf{Từ viết tắt} & \textbf{Ý nghĩa} \\
\hline
\endfirsthead
\hline
\textbf{STT} & \textbf{Từ viết tắt} & \textbf{Ý nghĩa} \\
\hline
\endhead
1 & API & Application Programming Interface - Giao diện lập trình ứng dụng \\
\hline
2 & BSON & Binary JSON - Định dạng nhị phân của JSON \\
\hline
3 & CAP & Consistency, Availability, Partition Tolerance - Định lý CAP \\
\hline
4 & CLI & Command Line Interface - Giao diện dòng lệnh \\
\hline
5 & CRUD & Create, Read, Update, Delete - Các thao tác cơ bản \\
\hline
6 & CSDL & Cơ sở dữ liệu \\
\hline
7 & CSS & Cascading Style Sheets - Ngôn ngữ định kiểu \\
\hline
8 & HA & High Availability - Tính sẵn sàng cao \\
\hline
9 & HTML & HyperText Markup Language - Ngôn ngữ đánh dấu \\
\hline
10 & HTTP(S) & HyperText Transfer Protocol (Secure) - Giao thức truyền tải \\
\hline
11 & JSON & JavaScript Object Notation - Định dạng trao đổi dữ liệu \\
\hline
12 & JWT & JSON Web Token - Token xác thực dạng JSON \\
\hline
13 & MVC & Model-View-Controller - Mô hình thiết kế \\
\hline
14 & NoSQL & Not Only SQL - Cơ sở dữ liệu phi quan hệ \\
\hline
15 & PHP & PHP: Hypertext Preprocessor - Ngôn ngữ lập trình web \\
\hline
16 & RBAC & Role-Based Access Control - Phân quyền theo vai trò \\
\hline
17 & REST & Representational State Transfer - Kiến trúc API \\
\hline
18 & SQL & Structured Query Language - Ngôn ngữ truy vấn \\
\hline
19 & TLS/SSL & Transport Layer Security/Secure Sockets Layer - Bảo mật tầng vận chuyển \\
\hline
20 & URL & Uniform Resource Locator - Địa chỉ tài nguyên \\
\hline
21 & GUI & Graphical User Interface - Giao diện đồ họa \\
\hline
22 & AJAX & Asynchronous JavaScript and XML - Kỹ thuật web không đồng bộ \\
\hline
\end{longtable}

\section{Mã nguồn quan trọng}

\subsection{JWTHelper.php - Xác thực JWT}

\begin{lstlisting}[language=PHP, caption=JWTHelper.php - Class xử lý JWT authentication]
<?php
require_once 'vendor/autoload.php';
use Firebase\JWT\JWT;
use Firebase\JWT\Key;

class JWTHelper {
    private static $secret_key = "elibrary_secret_key_2025";
    private static $issuer = "elibrary_system";
    private static $algorithm = 'HS256';

    public static function generateToken($userData) {
        $issuedAt = time();
        $expirationTime = $issuedAt + (24 * 60 * 60); // 24 hours

        $payload = [
            'iss' => self::$issuer,
            'iat' => $issuedAt,
            'exp' => $expirationTime,
            'nbf' => $issuedAt,
            'data' => [
                'id' => (string)$userData['_id'],
                'username' => $userData['username'],
                'role' => $userData['role'] ?? 'customer',
                'fullName' => $userData['fullName'] ?? ''
            ]
        ];

        return JWT::encode($payload, self::$secret_key, self::$algorithm);
    }

    public static function validateToken($token) {
        try {
            $decoded = JWT::decode($token, new Key(self::$secret_key, self::$algorithm));
            return (array)$decoded;
        } catch (Exception $e) {
            return null;
        }
    }

    public static function requireAuth() {
        $headers = getallheaders();
        $authHeader = $headers['Authorization'] ?? '';

        if (preg_match('/Bearer\s+(.*)$/i', $authHeader, $matches)) {
            $token = $matches[1];
            $decoded = self::validateToken($token);
            if ($decoded) {
                return (array)$decoded['data'];
            }
        }

        http_response_code(401);
        echo json_encode(['error' => 'Unauthorized']);
        exit;
    }
}
\end{lstlisting}

\subsection{Connection.php - Kết nối MongoDB Replica Set}

\begin{lstlisting}[language=PHP, caption=Connection.php - Kết nối với MongoDB Replica Set]
<?php
require 'vendor/autoload.php';
use MongoDB\Client;
use MongoDB\Driver\ReadPreference;
use MongoDB\Driver\WriteConcern;

$Database = "Nhasach";

// Connection Mode: 'standalone', 'replicaset', 'sharded'
$mode = 'replicaset';

switch ($mode) {
    case 'replicaset':
        $uri = "mongodb://mongo1:27017,mongo2:27018,mongo3:27019/?replicaSet=rs0";
        $options = [
            'readPreference' => 'primaryPreferred',
            'w' => 'majority',
            'journal' => true
        ];
        break;
    case 'sharded':
        $uri = "mongodb://mongos1:27017,mongos2:27018";
        $options = ['w' => 'majority'];
        break;
    default:
        $uri = "mongodb://localhost:27017";
        $options = [];
}

try {
    $conn = new Client($uri, $options);
    $db = $conn->$Database;
} catch (Exception $e) {
    die("MongoDB connection failed: " . $e->getMessage());
}
\end{lstlisting}

\subsection{docker-compose.yml - Cấu hình Docker}

\begin{lstlisting}[language=yaml, caption=docker-compose.yml - Cấu hình MongoDB Replica Set]
version: '3.8'
services:
  mongo1:
    image: mongo:7.0
    container_name: mongo1
    ports:
      - "27017:27017"
    volumes:
      - mongo1_data:/data/db
    command: mongod --replSet rs0 --bind_ip_all
    networks:
      - mongo-net

  mongo2:
    image: mongo:7.0
    container_name: mongo2
    ports:
      - "27018:27017"
    volumes:
      - mongo2_data:/data/db
    command: mongod --replSet rs0 --bind_ip_all
    networks:
      - mongo-net

  mongo3:
    image: mongo:7.0
    container_name: mongo3
    ports:
      - "27019:27017"
    volumes:
      - mongo3_data:/data/db
    command: mongod --replSet rs0 --bind_ip_all
    networks:
      - mongo-net

networks:
  mongo-net:
    driver: bridge

volumes:
  mongo1_data:
  mongo2_data:
  mongo3_data:
\end{lstlisting}

\subsection{start\_system.sh - Script khởi động hệ thống}

\begin{lstlisting}[language=bash, caption=start\_system.sh - Script khởi động toàn bộ hệ thống]
#!/bin/bash
echo "=== Starting e-Library Distributed System ==="

# 1. Start MongoDB Replica Set
echo "[1/4] Starting MongoDB containers..."
cd /Users/tuannghiat/Downloads/Final\ CSDLTT
docker-compose up -d

# Wait for MongoDB to be ready
echo "Waiting for MongoDB to initialize..."
sleep 10

# 2. Initialize Replica Set
echo "[2/4] Initializing Replica Set..."
docker exec mongo1 mongosh --eval '
rs.initiate({
    _id: "rs0",
    members: [
        { _id: 0, host: "mongo1:27017", priority: 2 },
        { _id: 1, host: "mongo2:27017", priority: 1 },
        { _id: 2, host: "mongo3:27017", priority: 1 }
    ]
})
'

sleep 5

# 3. Start PHP servers for each node
echo "[3/4] Starting PHP servers..."
php -S localhost:8001 -t Nhasach/ &
php -S localhost:8002 -t NhasachHaNoi/ &
php -S localhost:8003 -t NhasachDaNang/ &
php -S localhost:8004 -t NhasachHoChiMinh/ &

# 4. Display access URLs
echo "[4/4] System started successfully!"
echo ""
echo "=== Access URLs ==="
echo "Central Hub:  http://localhost:8001"
echo "Ha Noi:       http://localhost:8002"
echo "Da Nang:      http://localhost:8003"
echo "Ho Chi Minh:  http://localhost:8004"
echo ""
echo "MongoDB:      mongodb://localhost:27017"
echo "==================="
\end{lstlisting}

\section{Cấu trúc thư mục dự án}

\begin{lstlisting}[language=bash, caption=Cấu trúc thư mục của dự án]
Final CSDLTT/
|-- docker-compose.yml          # MongoDB Replica Set config
|-- start_system.sh             # Startup script
|-- CLAUDE.md                   # Project documentation
|-- PROJECT_STATUS.md           # Status tracking
|
|-- Nhasach/                    # Central Hub (Port 8001)
|   |-- Connection.php          # MongoDB connection
|   |-- JWTHelper.php           # JWT authentication
|   |-- init_indexes.php        # Database indexes
|   |-- createadmin.php         # Create admin user
|   |-- composer.json           # PHP dependencies
|   |-- vendor/                 # Composer packages
|   |-- php/                    # Main application
|   |   |-- trangchu.php        # Homepage
|   |   |-- dangnhap.php        # Login page
|   |   |-- danhsachsach.php    # Book list
|   |   |-- giohang.php         # Shopping cart
|   |   |-- quanlysach.php      # Book management
|   |   |-- quanlynguoidung.php # User management
|   |   `-- dashboard.php       # Statistics dashboard
|   |-- api/                    # REST API endpoints
|   |   |-- statistics.php      # Aggregation Pipeline
|   |   |-- mapreduce.php       # Map-Reduce operations
|   |   |-- books.php           # Book CRUD
|   |   |-- users.php           # User CRUD
|   |   `-- orders.php          # Order processing
|   |-- css/                    # Stylesheets
|   `-- js/                     # JavaScript files
|
|-- NhasachHaNoi/               # Ha Noi Branch (Port 8002)
|-- NhasachDaNang/              # Da Nang Branch (Port 8003)
|-- NhasachHoChiMinh/           # Ho Chi Minh Branch (Port 8004)
|
|-- Data MONGODB export .json/  # Data exports
|-- scripts/                    # Utility scripts
|   |-- benchmark_real.js       # Benchmark script
|   `-- init-replicaset.sh      # RS initialization
|
|-- screenshots/                # UI screenshots
`-- report_latex/               # LaTeX report files
\end{lstlisting}

\section{Thông tin đăng nhập hệ thống}

\begin{table}[H]
\centering
\caption{Thông tin đăng nhập các node}
\begin{tabular}{|l|l|l|l|l|}
\hline
\textbf{Node} & \textbf{Port} & \textbf{Admin} & \textbf{Customer} & \textbf{Password} \\
\hline
Central Hub & 8001 & admin & testcustomer & 123456 \\
\hline
Hà Nội & 8002 & adminHN & annv, tuannghia & 123456 \\
\hline
Đà Nẵng & 8003 & adminDN & linhhtt, phuongltt & 123456 \\
\hline
Hồ Chí Minh & 8004 & adminHCM & huynq, yennt & 123456 \\
\hline
\end{tabular}
\end{table}

\begin{table}[H]
\centering
\caption{Thông tin kết nối MongoDB}
\begin{tabular}{|l|l|}
\hline
\textbf{Thành phần} & \textbf{Thông tin} \\
\hline
Replica Set Name & rs0 \\
\hline
Primary & mongo1:27017 \\
\hline
Secondary 1 & mongo2:27018 \\
\hline
Secondary 2 & mongo3:27019 \\
\hline
Connection URI & mongodb://mongo1:27017,mongo2:27018,mongo3:27019/?replicaSet=rs0 \\
\hline
\end{tabular}
\end{table}

\section{Thông tin liên hệ và mã nguồn}

Toàn bộ mã nguồn của dự án được công khai trên GitHub, bao gồm hướng dẫn cài đặt chi tiết, cấu hình Docker, và dữ liệu mẫu để chạy thử nghiệm.

\begin{table}[H]
\centering
\caption{Thông tin liên hệ và tài nguyên dự án}
\begin{tabular}{|l|p{10cm}|}
\hline
\textbf{Thông tin} & \textbf{Chi tiết} \\
\hline
GitHub Repository & \url{https://github.com/TatcataiTTN/Final-CSDLTT-HNUE-distributed-elibrary} \\
\hline
Email liên hệ & truongtuannghia1248@gmail.com \\
\hline
Số điện thoại & 0973 958 574 \\
\hline
Trường & Đại học Sư phạm Hà Nội (HNUE) \\
\hline
Môn học & Cơ sở dữ liệu tiên tiến - Cao học K35 \\
\hline
\end{tabular}
\end{table}

\vspace{0.5cm}
\begin{center}
\fbox{\parbox{0.9\textwidth}{
\centering
\textbf{Hướng dẫn cài đặt nhanh}\\[0.3cm]
\begin{enumerate}[leftmargin=*]
    \item Clone repository: \texttt{git clone https://github.com/TatcataiTTN/...}
    \item Cài đặt MongoDB 8.0 và PHP 8.4 với MongoDB extension
    \item Chạy \texttt{composer install} trong mỗi thư mục node
    \item Import dữ liệu từ thư mục \texttt{Data MONGODB export .json}
    \item Khởi động PHP server: \texttt{php -S localhost:8001 -t Nhasach}
\end{enumerate}
\textit{Chi tiết đầy đủ có trong file README.md và README\_STARTUP.md của repository.}
}}\end{center}
