% ============================================================================
% CHAPTER 5: ATMOSPHERIC CHANNEL MODELS
% ============================================================================

\chapter{Atmospheric Channel Models}
\label{chap:channel_models}

\begin{quote}
\textit{This chapter reviews the literature on atmospheric channel modeling for free-space optical communications, with emphasis on turbulence models, the Gamma-Gamma distribution, and their application to satellite-based QKD systems.}
\end{quote}

% ============================================================================
% SECTION 5.1: FSO CHANNEL OVERVIEW
% ============================================================================

\section{Free-Space Optical Channel Characteristics}
\label{sec:fso_overview}

\subsection{Loss Mechanisms}

The satellite-to-ground FSO channel introduces multiple loss mechanisms that affect QKD performance:

\begin{equation}
h_{total} = h_l \cdot h_a \cdot h_s \cdot h_p \cdot h_t
\label{eq:total_channel}
\end{equation}

where:
\begin{itemize}
    \item $h_l$: Free-space path loss
    \item $h_a$: Atmospheric attenuation
    \item $h_s$: Beam spreading loss
    \item $h_p$: Pointing error loss
    \item $h_t$: Atmospheric turbulence fading
\end{itemize}

\subsection{Literature Foundation}

Kaushal and Kaddoum (2017) \cite{kaushal2017space_optical} provide a comprehensive survey of space optical communication challenges:

\textbf{Key Topics Covered:}
\begin{itemize}
    \item Atmospheric effects (absorption, scattering, turbulence)
    \item Mitigation techniques (adaptive optics, diversity)
    \item Link budget considerations
    \item Pointing, acquisition, and tracking
\end{itemize}

% ============================================================================
% SECTION 5.2: FREE-SPACE PATH LOSS
% ============================================================================

\section{Free-Space Path Loss}
\label{sec:path_loss}

\subsection{Inverse Square Law}

The geometric spreading of optical beams follows:

\begin{equation}
h_l = \left( \frac{\lambda}{4\pi D_{SG}} \right)^2
\label{eq:fspl}
\end{equation}

where $D_{SG}$ is the satellite-to-ground distance:

\begin{equation}
D_{SG} = \sqrt{(H_S - H_G)^2 \cdot \sec^2(\zeta) + 2R_E(H_S - H_G) \cdot \sec(\zeta)}
\label{eq:slant_range}
\end{equation}

\textbf{Typical Values (Nguyen et al. parameters):}
\begin{itemize}
    \item $H_S = 600$ km, $\zeta = 50°$: $D_{SG} \approx 783$ km
    \item Path loss: $\sim$260 dB (before telescope gains)
\end{itemize}

% ============================================================================
% SECTION 5.3: ATMOSPHERIC ATTENUATION
% ============================================================================

\section{Atmospheric Attenuation}
\label{sec:attenuation}

\subsection{Beer-Lambert Law}

Atmospheric attenuation follows exponential decay:

\begin{equation}
h_a = \exp\left( -\gamma \cdot \frac{H_\beta - H_G}{\cos(\zeta)} \right)
\label{eq:attenuation}
\end{equation}

where $\gamma$ is the attenuation coefficient (dB/km) and $H_\beta$ is the effective atmospheric height.

\subsection{Weather Dependence}

\begin{table}[h]
\centering
\caption{Atmospheric Attenuation Coefficients}
\label{tab:attenuation_coeffs}
\begin{tabular}{lcc}
\toprule
\textbf{Weather Condition} & \textbf{$\gamma$ (dB/km)} & \textbf{Visibility (km)} \\
\midrule
Very clear & 0.0 -- 0.2 & $>$50 \\
Clear & 0.2 -- 0.5 & 23 -- 50 \\
Light haze & 0.5 -- 1.0 & 10 -- 23 \\
Haze & 1.0 -- 2.0 & 4 -- 10 \\
Light rain & 2.0 -- 4.0 & 2 -- 4 \\
Heavy rain & $>$10 & $<$1 \\
\bottomrule
\end{tabular}
\end{table}

\textbf{Relevance:} Nguyen et al. uses $\gamma = 0.43$ dB/km (clear conditions) as baseline.

% ============================================================================
% SECTION 5.4: ATMOSPHERIC TURBULENCE
% ============================================================================

\section{Atmospheric Turbulence}
\label{sec:turbulence}

\subsection{Physical Origin}

Atmospheric turbulence arises from temperature variations causing refractive index fluctuations. These fluctuations are characterized by the refractive index structure parameter $C_n^2$, which varies with altitude, time, and location.

\subsection{Hufnagel-Valley Model}

The Hufnagel-Valley (H-V) turbulence profile models $C_n^2$ as a function of altitude \cite{alhabash2001gammagamma}:

\begin{equation}
C_n^2(h) = 0.00594 \left(\frac{w}{27}\right)^2 (10^{-5}h)^{10} e^{-h/1000} + 2.7 \times 10^{-16} e^{-h/1500} + C_n^2(0) e^{-h/100}
\label{eq:hufnagel_valley}
\end{equation}

where:
\begin{itemize}
    \item $w$: RMS wind speed (typically 21 m/s)
    \item $h$: Altitude in meters
    \item $C_n^2(0)$: Ground-level turbulence strength
\end{itemize}

\textbf{Turbulence Regimes:}
\begin{itemize}
    \item Weak: $C_n^2(0) = 5 \times 10^{-15}$ m$^{-2/3}$
    \item Strong: $C_n^2(0) = 7 \times 10^{-12}$ m$^{-2/3}$
\end{itemize}

\subsection{Scintillation Index}

The Rytov variance characterizes scintillation strength:

\begin{equation}
\sigma_R^2 = 2.25 k^{7/6} \sec^{11/6}(\zeta) \int_{H_G}^{H_S} C_n^2(h) \left(1 - \frac{h-H_G}{H_S-H_G}\right)^{5/6} (h-H_G)^{5/6} dh
\label{eq:rytov}
\end{equation}

% ============================================================================
% SECTION 5.5: GAMMA-GAMMA DISTRIBUTION
% ============================================================================

\section{Gamma-Gamma Turbulence Model}
\label{sec:gamma_gamma}

\subsection{Al-Habash et al. (2001)}

Al-Habash, Andrews, and Phillips \cite{alhabash2001gammagamma} developed the Gamma-Gamma (GG) distribution for modeling irradiance fluctuations in moderate-to-strong turbulence:

\begin{equation}
f_{h_t}(h_t) = \frac{2(\alpha\beta)^{(\alpha+\beta)/2}}{\Gamma(\alpha)\Gamma(\beta)} h_t^{(\alpha+\beta)/2-1} K_{\alpha-\beta}\left(2\sqrt{\alpha\beta h_t}\right)
\label{eq:gamma_gamma_pdf}
\end{equation}

where:
\begin{itemize}
    \item $\alpha, \beta$: Large-scale and small-scale scintillation parameters
    \item $K_{\nu}(\cdot)$: Modified Bessel function of the second kind
    \item $\Gamma(\cdot)$: Gamma function
\end{itemize}

\subsection{Scintillation Parameters}

For spherical wave propagation (satellite downlink):

\begin{align}
\alpha &= \left[ \exp\left( \frac{0.49\sigma_R^2}{(1 + 1.11\sigma_R^{12/5})^{7/6}} \right) - 1 \right]^{-1} \label{eq:alpha}\\
\beta &= \left[ \exp\left( \frac{0.51\sigma_R^2}{(1 + 0.69\sigma_R^{12/5})^{5/6}} \right) - 1 \right]^{-1} \label{eq:beta}
\end{align}

\subsection{Physical Interpretation}

The GG model treats turbulence as a multiplicative process:

\begin{equation}
h_t = h_X \cdot h_Y
\label{eq:multiplicative}
\end{equation}

where $h_X \sim \text{Gamma}(\alpha, 1/\alpha)$ (large eddies) and $h_Y \sim \text{Gamma}(\beta, 1/\beta)$ (small eddies).

\subsection{Vasylyev et al. (2016)}

Vasylyev, Semenov, and Vogel \cite{vasylyev2016atmospheric_channels} extended atmospheric channel modeling specifically for quantum communications:

\textbf{Key Contributions:}
\begin{itemize}
    \item Quantum-specific treatment of channel loss
    \item Entanglement preservation analysis through turbulent channels
    \item Elliptic beam approximation for realistic beam shapes
\end{itemize}

% ============================================================================
% SECTION 5.6: BEAM EFFECTS
% ============================================================================

\section{Beam Spreading and Wandering}
\label{sec:beam_effects}

\subsection{Liorni et al. (2019)}

Liorni, Kampermann, and Bru{\ss} \cite{liorni2019atmospheric_characteristics} analyzed beam effects on satellite QKD:

\textbf{Topics Covered:}
\begin{itemize}
    \item Diffraction-limited beam spreading
    \item Turbulence-induced beam wandering
    \item Combined pointing and tracking effects
    \item Weather-dependent performance
\end{itemize}

\subsection{Effective Beam Width}

The beam width at the receiver includes diffraction and turbulence contributions:

\begin{equation}
\omega_{eq}^2 = \omega_D^2 \left(1 + \frac{D_{SG}^2}{k^2 \omega_0^4} + 1.33\sigma_R^2 \Lambda^{5/6} \right)
\label{eq:effective_beam}
\end{equation}

\subsection{Ma et al. (2015)}

Ma et al. \cite{ma2015satellite_downlink} analyzed satellite-to-ground coherent optical communications with spatial diversity:

\textbf{Relevance to Nguyen et al.:}
\begin{itemize}
    \item Validated GG distribution for satellite downlinks
    \item Provided framework for heterodyne detection analysis
    \item Demonstrated spatial diversity benefits
\end{itemize}

% ============================================================================
% SECTION 5.7: CHANNEL MODEL IN NGUYEN ET AL.
% ============================================================================

\section{Channel Model Implementation in Nguyen et al.}
\label{sec:nguyen_channel}

\subsection{Combined Channel Coefficient}

Nguyen et al. combines all effects into a single channel coefficient:

\begin{equation}
h = h_l \cdot h_a \cdot h_s \cdot h_t
\label{eq:nguyen_channel}
\end{equation}

\subsection{Specific Parameter Values}

\begin{table}[h]
\centering
\caption{Channel Parameters in Nguyen et al. (2021)}
\label{tab:nguyen_channel_params}
\begin{tabular}{lll}
\toprule
\textbf{Parameter} & \textbf{Symbol} & \textbf{Value} \\
\midrule
Wavelength & $\lambda$ & 1550 nm \\
Satellite altitude & $H_S$ & 600 km \\
Ground station height & $H_G$ & 5 m \\
Atmospheric height & $H_\beta$ & 20 km \\
Zenith angle & $\zeta$ & 50° \\
Wind speed & $w$ & 21 m/s \\
Beam width at ground & $\omega_D$ & 50 m \\
Attenuation coefficient & $\gamma$ & 0.43 dB/km \\
\midrule
\multicolumn{3}{l}{\textit{Weak Turbulence}} \\
$C_n^2(0)$ & & $5 \times 10^{-15}$ m$^{-2/3}$ \\
$\alpha$ & & Calculated via Eq. \ref{eq:alpha} \\
$\beta$ & & Calculated via Eq. \ref{eq:beta} \\
\midrule
\multicolumn{3}{l}{\textit{Strong Turbulence}} \\
$C_n^2(0)$ & & $7 \times 10^{-12}$ m$^{-2/3}$ \\
\bottomrule
\end{tabular}
\end{table}

\subsection{Integration with QBER Analysis}

The turbulence-affected channel coefficient determines received power:

\begin{equation}
P_R = P_T \cdot G_T \cdot G_R \cdot h
\label{eq:received_power}
\end{equation}

This feeds into SNR calculation and subsequently QBER analysis.

% ============================================================================
% SECTION 5.8: RECENT DEVELOPMENTS
% ============================================================================

\section{Recent Developments (2020-2025)}
\label{sec:recent_channel}

\subsection{Fisher-Snedecor F Distribution}

Recent experimental data suggest the F distribution may provide better fit across weak-to-strong turbulence than Gamma-Gamma in some scenarios.

\subsection{Machine Learning Approaches}

Emerging research applies ML for:
\begin{itemize}
    \item Channel state prediction
    \item Adaptive parameter optimization
    \item Turbulence mitigation
\end{itemize}

\subsection{Tropical Atmosphere Considerations}

For Vietnam and similar regions, specialized modeling may be needed:
\begin{itemize}
    \item Higher humidity effects
    \item Monsoon season variability
    \item Aerosol loading differences
\end{itemize}

% ============================================================================
% SECTION 5.9: CHAPTER SUMMARY
% ============================================================================

\section{Chapter Summary}
\label{sec:ch5_summary}

This chapter reviewed atmospheric channel modeling literature relevant to Nguyen et al. (2021):

\begin{enumerate}
    \item \textbf{Hufnagel-Valley Model:} Standard altitude-dependent turbulence profile
    \item \textbf{Gamma-Gamma Distribution:} Primary model for moderate-to-strong turbulence
    \item \textbf{Scintillation Parameters:} $\alpha$, $\beta$ derived from Rytov variance
    \item \textbf{Beam Effects:} Spreading and wandering impact on received power
    \item \textbf{Weather Dependence:} Attenuation varies significantly with conditions
\end{enumerate}

\textbf{Key Insight:} Nguyen et al. adopts well-established atmospheric models (GG distribution, H-V profile) providing solid foundation for QBER and KLR analysis.
