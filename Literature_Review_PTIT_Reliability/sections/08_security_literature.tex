% ============================================================================
% CHAPTER 8: SECURITY ANALYSIS LITERATURE
% ============================================================================

\chapter{Security Analysis Literature}
\label{chap:security_literature}

\begin{quote}
\textit{This chapter reviews the security analysis literature for practical QKD systems, covering attack models, security proofs, finite-key analysis, and device imperfections.}
\end{quote}

% ============================================================================
% SECTION 8.1: SECURITY FRAMEWORK
% ============================================================================

\section{QKD Security Framework}
\label{sec:security_framework}

\subsection{Information-Theoretic Security}

QKD provides security based on physical laws rather than computational assumptions:

\begin{itemize}
    \item \textbf{No-Cloning:} Quantum states cannot be perfectly copied
    \item \textbf{Measurement Disturbance:} Eavesdropping creates detectable errors
    \item \textbf{Unconditional Security:} Secure against unlimited computational power
\end{itemize}

\subsection{Security Hierarchy}

\begin{table}[h]
\centering
\caption{Attack Classification Hierarchy}
\label{tab:attack_hierarchy}
\begin{tabular}{lcc}
\toprule
\textbf{Attack Type} & \textbf{Power} & \textbf{QBER Threshold} \\
\midrule
Individual attacks & Weakest & 14.6\% \\
Collective attacks & Medium & 11.0\% \\
Coherent attacks & Strongest & 7.1\% \\
\bottomrule
\end{tabular}
\end{table}

% ============================================================================
% SECTION 8.2: SCARANI ET AL. (2009)
% ============================================================================

\section{Practical Security Framework}
\label{sec:scarani}

\subsection{Scarani et al. (2009)}

``The Security of Practical Quantum Key Distribution'' \cite{scarani2009security} established the definitive framework for analyzing practical QKD security:

\textbf{Key Contributions:}
\begin{enumerate}
    \item \textbf{Device Imperfection Modeling:}
    \begin{itemize}
        \item Non-ideal single-photon sources
        \item Detector inefficiencies and dark counts
        \item Channel losses and noise
    \end{itemize}

    \item \textbf{Attack Analysis:}
    \begin{itemize}
        \item Photon-number-splitting attacks
        \item Intercept-resend strategies
        \item Trojan horse attacks
    \end{itemize}

    \item \textbf{Security Parameter Calculation:}
    \begin{itemize}
        \item QBER threshold derivation
        \item Privacy amplification requirements
        \item Finite-key corrections
    \end{itemize}
\end{enumerate}

\subsection{Relevance to Nguyen et al.}

Scarani's framework provides:
\begin{itemize}
    \item QBER threshold ($10^{-3}$) justification
    \item Security analysis methodology
    \item Foundation for URA attack analysis
\end{itemize}

% ============================================================================
% SECTION 8.3: FINITE-KEY ANALYSIS
% ============================================================================

\section{Finite-Key Security}
\label{sec:finite_key}

\subsection{The Finite-Key Challenge}

Asymptotic security proofs assume infinite key length. Practical systems require finite-key analysis:

\begin{equation}
l_{secure} = l_{sifted} - \text{leak}_{EC} - \text{leak}_{PE} - \text{PA}
\label{eq:finite_key}
\end{equation}

where:
\begin{itemize}
    \item $\text{leak}_{EC}$: Information leaked during error correction
    \item $\text{leak}_{PE}$: Information leaked during parameter estimation
    \item $\text{PA}$: Privacy amplification compression
\end{itemize}

\subsection{Tomamichel et al. (2012)}

``Tight Finite-Key Analysis for Quantum Cryptography'' \cite{tomamichel2012finite_resources} in Nature Communications:

\textbf{Key Contributions:}
\begin{itemize}
    \item Composable security definitions
    \item Tight bounds on finite-size effects
    \item Practical parameter optimization
\end{itemize}

\subsection{Impact on Key Rate}

Finite-key effects become significant for:
\begin{itemize}
    \item Short satellite passes (limited transmission time)
    \item Low repetition rate systems
    \item High-loss channels
\end{itemize}

\textbf{Gap in Nguyen et al.:} The paper uses asymptotic analysis; finite-key effects not incorporated.

% ============================================================================
% SECTION 8.4: DEVICE SECURITY
% ============================================================================

\section{Realistic Device Security}
\label{sec:device_security}

\subsection{Xu et al. (2020)}

``Secure Quantum Key Distribution with Realistic Devices'' \cite{xu2020realistic_devices} in Reviews of Modern Physics:

\textbf{Coverage:}
\begin{enumerate}
    \item \textbf{Source Imperfections:}
    \begin{itemize}
        \item Multi-photon emission
        \item State preparation flaws
        \item Side-channel leakage
    \end{itemize}

    \item \textbf{Detector Vulnerabilities:}
    \begin{itemize}
        \item Blinding attacks
        \item Time-shift attacks
        \item Efficiency mismatch
    \end{itemize}

    \item \textbf{Countermeasures:}
    \begin{itemize}
        \item Decoy states for source issues
        \item MDI-QKD for detector immunity
        \item Device characterization protocols
    \end{itemize}
\end{enumerate}

\subsection{Measurement-Device-Independent QKD}

MDI-QKD removes all detector side-channel attacks:
\begin{itemize}
    \item Bell state measurement at untrusted node
    \item Security independent of detector imperfections
    \item Demonstrated over 500 km fiber
\end{itemize}

% ============================================================================
% SECTION 8.5: CV-QKD SECURITY
% ============================================================================

\section{CV-QKD Security}
\label{sec:cvqkd_security}

\subsection{Leverrier (2015)}

``Composable Security Proof for CV-QKD'' \cite{leverrier2015cv_security}:

\textbf{Contribution:} First composable security proof for Gaussian-modulated CV-QKD with coherent states against collective attacks.

\subsection{Discrete Modulation Security}

For QPSK-like discrete modulation:
\begin{itemize}
    \item Security proofs developed post-2018
    \item No longer requires linear channel assumption
    \item Bounded via uncertainty principle methods
\end{itemize}

\textbf{Relevance:} Nguyen et al.'s QPSK approach benefits from these developments.

% ============================================================================
% SECTION 8.6: NGUYEN ET AL. SECURITY ANALYSIS
% ============================================================================

\section{Security Analysis in Nguyen et al.}
\label{sec:nguyen_security}

\subsection{Unauthorized Receiver Attack (URA)}

Nguyen et al. analyzes security against an eavesdropper (Eve) positioned near Bob:

\textbf{Attack Model:}
\begin{itemize}
    \item Eve places receiver within satellite beam footprint
    \item Bob at beam center ($r = 0$)
    \item Eve at distance $D_{E-B}$ from Bob
\end{itemize}

\subsection{Eve's Received Power}

Eve's power decreases with distance from beam center:

\begin{equation}
P_{R,Eve} \propto A_0 \exp\left(-\frac{2D_{E-B}^2}{\omega_{Deq}^2}\right)
\label{eq:eve_power}
\end{equation}

\subsection{Security Boundary}

\begin{table}[h]
\centering
\caption{Eve's QBER vs. Distance}
\label{tab:eve_qber}
\begin{tabular}{lccc}
\toprule
\textbf{$D_{E-B}$ (m)} & \textbf{Weak Turb.} & \textbf{Strong Turb.} & \textbf{Security} \\
\midrule
0 & Same as Bob & Same as Bob & Compromised \\
10 & $\sim 10^{-3}$ & $\sim 10^{-2}$ & Marginal \\
20 & $\sim 10^{-2}$ & $\sim 10^{-2}$ & Marginal \\
\textbf{30} & $\mathbf{> 10^{-2}}$ & $\mathbf{> 10^{-2}}$ & \textbf{Secure} \\
50+ & $> 10^{-1}$ & $> 10^{-1}$ & Secure \\
\bottomrule
\end{tabular}
\end{table}

\textbf{Conclusion:} $D_{E-B} > 30$ m ensures Eve cannot obtain usable key.

\subsection{Limitations of Analysis}

\begin{enumerate}
    \item \textbf{URA Only:} Does not consider intercept-resend or other attacks
    \item \textbf{Asymptotic:} Finite-key effects not analyzed
    \item \textbf{Collective Attacks:} General attack security not proven
\end{enumerate}

% ============================================================================
% SECTION 8.7: RECENT DEVELOPMENTS
% ============================================================================

\section{Recent Security Developments (2022-2025)}
\label{sec:recent_security}

\subsection{Numerical Security Proofs}

Physical Review Research (2022) presents numerical security proofs for:
\begin{itemize}
    \item Decoy-state BB84 with basis misalignment
    \item MDI-QKD practical implementations
    \item Fine-grained statistics utilization
\end{itemize}

\subsection{Finite-Key for Heterodyne (2025)}

arXiv:2501.10278 addresses:
\begin{itemize}
    \item Phase imbalance in heterodyne detection
    \item Practical finite-key bounds
    \item System optimization under realistic conditions
\end{itemize}

\subsection{Post-Quantum Considerations}

While QKD is quantum-safe, practical considerations include:
\begin{itemize}
    \item Authentication channel security
    \item Key management infrastructure
    \item Hybrid classical-quantum systems
\end{itemize}

% ============================================================================
% SECTION 8.8: CHAPTER SUMMARY
% ============================================================================

\section{Chapter Summary}
\label{sec:ch8_summary}

This chapter reviewed security analysis literature:

\begin{enumerate}
    \item \textbf{Foundational Framework:} Scarani et al. established practical security analysis
    \item \textbf{Finite-Key:} Tomamichel et al. provided composable security with finite resources
    \item \textbf{Device Security:} Xu et al. addressed realistic device imperfections
    \item \textbf{CV-QKD:} Leverrier proved composable security for coherent detection
    \item \textbf{Nguyen et al.:} Analyzed URA scenario; gaps remain in finite-key and general attacks
\end{enumerate}

\textbf{Key Gap:} Nguyen et al. provides important URA analysis but would benefit from finite-key analysis and security proof against general attacks.
