% ============================================================================
% CHAPTER 9: COMPARATIVE ANALYSIS
% ============================================================================

\chapter{Comparative Analysis}
\label{chap:comparative}

\begin{quote}
\textit{This chapter synthesizes the literature reviewed in previous chapters, positioning Nguyen et al. (2021) within the broader context of satellite QKD research and identifying its unique contributions.}
\end{quote}

% ============================================================================
% SECTION 9.1: METHODOLOGY COMPARISON
% ============================================================================

\section{Methodology Comparison}
\label{sec:methodology_comparison}

\subsection{Detection Approach Comparison}

\begin{table}[h]
\centering
\caption{Detection Approaches in Satellite QKD Literature}
\label{tab:detection_comparison}
\small
\begin{tabular}{lcccc}
\toprule
\textbf{System} & \textbf{Detection} & \textbf{Wavelength} & \textbf{Sensitivity} & \textbf{Complexity} \\
\midrule
Micius (2017) & SPD (Si-APD) & 850 nm & High & High \\
CV-QKD (Dequal) & Coherent & 1550 nm & Medium & Medium \\
DT/DD (Trinh) & Direct & 1550 nm & Low & Low \\
\textbf{DT/HD (Nguyen)} & \textbf{Heterodyne} & \textbf{1550 nm} & \textbf{Very High} & \textbf{Medium} \\
\bottomrule
\end{tabular}
\end{table}

\subsection{Protocol Comparison}

\begin{table}[h]
\centering
\caption{Protocol Approaches Comparison}
\label{tab:protocol_comparison}
\begin{tabular}{lccc}
\toprule
\textbf{Approach} & \textbf{Modulation} & \textbf{States} & \textbf{Security Basis} \\
\midrule
BB84 (Micius) & Polarization & 4 & BB84 proof \\
Gaussian CV-QKD & Gaussian & $\infty$ & CV proof \\
Discrete CV-QKD & QPSK/8PSK & 4/8 & CV proof \\
\textbf{QPSK-BB84 (Nguyen)} & \textbf{Phase} & \textbf{4} & \textbf{BB84 mapping} \\
\bottomrule
\end{tabular}
\end{table}

\subsection{Error Handling Comparison}

\begin{table}[h]
\centering
\caption{Error Handling Approaches}
\label{tab:error_comparison}
\begin{tabular}{lcccc}
\toprule
\textbf{Method} & \textbf{Type} & \textbf{Overhead} & \textbf{Adaptivity} & \textbf{Implementation} \\
\midrule
CASCADE & Interactive & High & Low & Complex \\
LDPC & FEC & Medium & Low & Complex \\
Polar & FEC & Medium & Low & Medium \\
\textbf{ARQ (Nguyen)} & \textbf{Retransmit} & \textbf{Low} & \textbf{High} & \textbf{Simple} \\
\bottomrule
\end{tabular}
\end{table}

% ============================================================================
% SECTION 9.2: PERFORMANCE COMPARISON
% ============================================================================

\section{Performance Comparison}
\label{sec:performance_comparison}

\subsection{Power Requirements}

\begin{table}[h]
\centering
\caption{Transmitted Power Comparison (QBER $\leq 10^{-3}$)}
\label{tab:power_comparison}
\begin{tabular}{lcc}
\toprule
\textbf{Scheme} & \textbf{Required $P_T$} & \textbf{Relative Gain} \\
\midrule
SIM/BPSK-DT (Baseline) & 45 dBm & 0 dB \\
QPSK-DT/DD (Trinh 2018) & 35 dBm & 10 dB \\
\textbf{QPSK-DT/HD (Nguyen 2021)} & \textbf{25 dBm} & \textbf{20 dB} \\
\bottomrule
\end{tabular}
\end{table}

\subsection{Key Rate Comparison}

\begin{table}[h]
\centering
\caption{Key Rate Performance Comparison}
\label{tab:key_rate_comparison}
\small
\begin{tabular}{lccc}
\toprule
\textbf{System} & \textbf{Distance} & \textbf{Key Rate} & \textbf{Status} \\
\midrule
Micius (2017) & 530 km & 40.2 kbps sifted & Experimental \\
Micius (2017) & 1034 km & 1.2 kbps sifted & Experimental \\
Chen (2021) & 4600 km & 47.8 kbps & Experimental \\
\textbf{Nguyen (2021)} & \textbf{600 km (LEO)} & \textbf{Simulation} & \textbf{Theoretical} \\
\bottomrule
\end{tabular}
\end{table}

\subsection{Reliability Comparison}

\begin{table}[h]
\centering
\caption{Reliability Metrics Comparison}
\label{tab:reliability_comparison}
\begin{tabular}{lccc}
\toprule
\textbf{System} & \textbf{Error Handling} & \textbf{Reliability Metric} & \textbf{Value} \\
\midrule
Micius & LDPC FEC & QBER & 1.1--3.2\% \\
Chen Network & Trusted relays & Network availability & $>$99\% \\
\textbf{Nguyen} & \textbf{ARQ ($M=4$)} & \textbf{KLR} & $\mathbf{<10^{-4}}$ \\
\bottomrule
\end{tabular}
\end{table}

% ============================================================================
% SECTION 9.3: UNIQUE CONTRIBUTIONS
% ============================================================================

\section{Unique Contributions of Nguyen et al.}
\label{sec:unique_contributions}

\subsection{Novel Elements}

\begin{enumerate}
    \item \textbf{First ARQ for Satellite QKD}
    \begin{itemize}
        \item No prior literature applies retransmission to quantum key distribution
        \item Paradigm shift from error correction to error avoidance
        \item Enabled by classical feedback channel availability
    \end{itemize}

    \item \textbf{3-D Markov Chain Model}
    \begin{itemize}
        \item Novel state space: (buffer, channel, retransmission)
        \item Analytical KLR calculation capability
        \item Optimization framework without extensive simulation
    \end{itemize}

    \item \textbf{DT/HD Integration}
    \begin{itemize}
        \item First combination of dual-threshold with heterodyne for QKD
        \item 20 dB improvement over previous PTIT work
        \item Bridges DV and CV detection paradigms
    \end{itemize}

    \item \textbf{Cross-Layer Design}
    \begin{itemize}
        \item Physical layer: QPSK + DT/HD
        \item Link layer: ARQ retransmission
        \item Integrated optimization approach
    \end{itemize}
\end{enumerate}

\subsection{Research Gaps Addressed}

\begin{table}[h]
\centering
\caption{Research Gaps Addressed}
\label{tab:gaps_addressed}
\begin{tabular}{lcc}
\toprule
\textbf{Gap in Literature} & \textbf{Addressed?} & \textbf{How} \\
\midrule
Reliability without FEC overhead & \cmark & ARQ scheme \\
Link layer QKD analysis & \cmark & 3-D Markov model \\
Coherent detection for BB84-like & \cmark & QPSK + heterodyne \\
KLR metric formalization & \cmark & Analytical derivation \\
Vietnamese satellite QKD research & \cmark & PTIT contribution \\
\bottomrule
\end{tabular}
\end{table}

% ============================================================================
% SECTION 9.4: PTIT RESEARCH EVOLUTION
% ============================================================================

\section{PTIT Research Evolution}
\label{sec:ptit_evolution}

\subsection{Research Timeline}

\begin{table}[h]
\centering
\caption{PTIT Satellite QKD Research Timeline}
\label{tab:ptit_timeline}
\small
\begin{tabular}{clll}
\toprule
\textbf{Year} & \textbf{Paper} & \textbf{Contribution} & \textbf{Advancement} \\
\midrule
2018 & Trinh et al. & DT/DD for QKD & Introduced DT concept \\
2019 & Vu et al. & HAP-aided relay & Extended to HAP \\
\textbf{2021} & \textbf{Nguyen et al.} & \textbf{DT/HD + ARQ} & \textbf{+20 dB + reliability} \\
2023 & Nguyen et al. & CV-QKD extension & Extended to CV-QKD \\
2023 & Vu et al. & Network coding & Multi-satellite \\
\bottomrule
\end{tabular}
\end{table}

\subsection{Cumulative Contributions}

The PTIT research group has systematically:
\begin{enumerate}
    \item Introduced dual-threshold detection for QKD
    \item Improved sensitivity through heterodyne detection
    \item Added reliability through retransmission
    \item Extended to CV-QKD protocols
    \item Explored network-level optimizations
\end{enumerate}

% ============================================================================
% SECTION 9.5: INTERNATIONAL CONTEXT
% ============================================================================

\section{International Context}
\label{sec:international_context}

\subsection{Geographic Distribution of Research}

\begin{table}[h]
\centering
\caption{Satellite QKD Research by Region}
\label{tab:geographic}
\begin{tabular}{lll}
\toprule
\textbf{Region} & \textbf{Focus} & \textbf{Key Institutions} \\
\midrule
China & Experimental demonstration & USTC, CAS \\
Europe & Mission planning & ESA, DLR, CNES \\
Japan & Ground experiments & NICT, JAXA \\
Canada & LEO development & Waterloo, CSA \\
Singapore & CubeSat QKD & NUS, CQT \\
\textbf{Vietnam} & \textbf{Theoretical analysis} & \textbf{PTIT, VAST} \\
\bottomrule
\end{tabular}
\end{table}

\subsection{Vietnamese Position}

Vietnam's contribution through PTIT:
\begin{itemize}
    \item Theoretical foundations for practical systems
    \item Novel detection and reliability approaches
    \item Regional capacity building
    \item Potential foundation for future missions
\end{itemize}

% ============================================================================
% SECTION 9.6: STRENGTHS AND LIMITATIONS
% ============================================================================

\section{Critical Assessment}
\label{sec:critical_assessment}

\subsection{Strengths}

\begin{enumerate}
    \item \textbf{Novel Integration:} First combination of QPSK, DT/HD, and ARQ
    \item \textbf{Analytical Rigor:} Comprehensive mathematical framework
    \item \textbf{Practical Focus:} Realistic system parameters
    \item \textbf{Significant Improvement:} Quantifiable 20 dB and $>$1000$\times$ gains
    \item \textbf{Regional Contribution:} Advances Vietnamese research capability
\end{enumerate}

\subsection{Limitations}

\begin{enumerate}
    \item \textbf{Simulation Only:} No experimental validation
    \item \textbf{Idealized Assumptions:}
    \begin{itemize}
        \item Perfect pointing and tracking
        \item No phase noise in local oscillator
        \item Ideal modulator/demodulator
    \end{itemize}
    \item \textbf{Security Gaps:}
    \begin{itemize}
        \item Asymptotic analysis only
        \item Limited attack model (URA)
        \item No composable security proof
    \end{itemize}
    \item \textbf{System Gaps:}
    \begin{itemize}
        \item Single satellite link
        \item No handover consideration
        \item No network integration
    \end{itemize}
\end{enumerate}

% ============================================================================
% SECTION 9.7: CHAPTER SUMMARY
% ============================================================================

\section{Chapter Summary}
\label{sec:ch9_summary}

This comparative analysis establishes:

\begin{enumerate}
    \item \textbf{Unique Position:} Nguyen et al. occupies a unique niche combining coherent detection with retransmission reliability
    \item \textbf{Performance Gains:} 20 dB power improvement and $>$1000$\times$ KLR reduction are significant
    \item \textbf{Research Gap Filling:} Addresses previously unexplored link-layer reliability for satellite QKD
    \item \textbf{Foundation Building:} Provides theoretical foundation for future Vietnamese quantum satellite missions
    \item \textbf{Remaining Work:} Experimental validation and security analysis extensions needed
\end{enumerate}
