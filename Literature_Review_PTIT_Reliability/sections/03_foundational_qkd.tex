% ============================================================================
% CHAPTER 3: FOUNDATIONAL QKD LITERATURE
% ============================================================================

\chapter{Foundational QKD Literature}
\label{chap:foundational_qkd}

\begin{quote}
\textit{This chapter reviews the foundational literature in Quantum Key Distribution, from the original BB84 protocol to modern continuous-variable approaches, establishing the theoretical basis for satellite-based QKD systems.}
\end{quote}

% ============================================================================
% SECTION 3.1: THE BB84 PROTOCOL
% ============================================================================

\section{The BB84 Protocol}
\label{sec:bb84}

\subsection{Historical Context}

The BB84 protocol, proposed by Charles Bennett and Gilles Brassard in 1984 \cite{bennett1984bb84}, represents the foundational breakthrough in quantum cryptography. Published at the IEEE International Conference on Computers, Systems and Signal Processing in Bangalore, India, this work established the principles that underpin all subsequent QKD protocols.

\textbf{Key Innovation:} BB84 was the first protocol to demonstrate that quantum mechanical principles---specifically the no-cloning theorem and the disturbance caused by measurement---could be exploited to achieve information-theoretically secure key distribution.

\subsection{Protocol Description}

The BB84 protocol operates using four quantum states organized into two conjugate bases:

\begin{table}[h]
\centering
\caption{BB84 Quantum States}
\label{tab:bb84_states}
\begin{tabular}{cccc}
\toprule
\textbf{Basis} & \textbf{Bit 0} & \textbf{Bit 1} & \textbf{Representation} \\
\midrule
Rectilinear ($+$) & $\ket{0}$ & $\ket{1}$ & Horizontal/Vertical \\
Diagonal ($\times$) & $\ket{+}$ & $\ket{-}$ & $\pm 45°$ Diagonal \\
\bottomrule
\end{tabular}
\end{table}

\textbf{Protocol Steps:}
\begin{enumerate}
    \item \textbf{Preparation:} Alice randomly chooses a bit value and a basis, prepares the corresponding quantum state
    \item \textbf{Transmission:} State is sent through quantum channel to Bob
    \item \textbf{Measurement:} Bob randomly chooses a measurement basis
    \item \textbf{Sifting:} Alice and Bob publicly compare bases; keep only matching-basis results
    \item \textbf{Error Estimation:} Sample subset to estimate QBER
    \item \textbf{Error Correction:} Correct remaining errors using classical protocols
    \item \textbf{Privacy Amplification:} Reduce Eve's potential information
\end{enumerate}

\subsection{Security Foundation}

The security of BB84 rests on fundamental quantum mechanical principles:

\begin{enumerate}
    \item \textbf{No-Cloning Theorem:} An unknown quantum state cannot be perfectly copied
    \item \textbf{Measurement Disturbance:} Measuring a quantum state in the wrong basis causes irreversible disturbance
    \item \textbf{Uncertainty Principle:} Conjugate observables cannot be simultaneously known with arbitrary precision
\end{enumerate}

\textbf{QBER Threshold:} The protocol remains secure when QBER $< 11\%$ for individual attacks, or QBER $< 7.1\%$ for coherent attacks.

\subsection{Relevance to Nguyen et al. (2021)}

Nguyen et al. maps the BB84 four-state structure to QPSK phase states:

\begin{equation}
\text{BB84 States} \rightarrow \text{QPSK Phases: } \phi_A \in \left\{ \frac{\pi}{4}, \frac{3\pi}{4}, \frac{5\pi}{4}, -\frac{\pi}{4} \right\}
\label{eq:bb84_qpsk_mapping}
\end{equation}

This mapping preserves the fundamental security properties while enabling coherent detection.

% ============================================================================
% SECTION 3.2: E91 PROTOCOL
% ============================================================================

\section{The E91 Protocol}
\label{sec:e91}

\subsection{Entanglement-Based QKD}

Artur Ekert proposed the E91 protocol in 1991 \cite{ekert1991e91}, introducing entanglement as the basis for QKD security. This protocol uses maximally entangled photon pairs (Bell states):

\begin{equation}
\ket{\Phi^+} = \frac{1}{\sqrt{2}}(\ket{00} + \ket{11})
\label{eq:bell_state}
\end{equation}

\subsection{Security via Bell Inequality}

E91's security is verified through violation of Bell's inequality:

\begin{equation}
S = |E(a,b) - E(a,b') + E(a',b) + E(a',b')| \leq 2 \text{ (Classical)}
\label{eq:bell_inequality}
\end{equation}

For maximally entangled states: $S = 2\sqrt{2} \approx 2.83$, proving quantum correlations.

\subsection{Advantages and Challenges}

\begin{table}[h]
\centering
\caption{E91 vs. BB84 Comparison}
\label{tab:e91_vs_bb84}
\begin{tabular}{lcc}
\toprule
\textbf{Aspect} & \textbf{BB84} & \textbf{E91} \\
\midrule
Source & Single photon/WCP & Entangled pairs \\
Security verification & QBER estimation & Bell inequality \\
Implementation & Simpler & More complex \\
Device-independence & No & Possible \\
Satellite demonstration & Micius (2017) & Micius (2017) \\
\bottomrule
\end{tabular}
\end{table}

% ============================================================================
% SECTION 3.3: CV-QKD
% ============================================================================

\section{Continuous-Variable QKD}
\label{sec:cvqkd}

\subsection{Paradigm Shift}

Continuous-Variable QKD (CV-QKD), pioneered by Grosshans et al. (2003) \cite{grosshans2003cv_qkd}, represents a fundamental departure from discrete-variable approaches:

\begin{table}[h]
\centering
\caption{DV-QKD vs. CV-QKD}
\label{tab:dv_cv_comparison}
\begin{tabular}{lcc}
\toprule
\textbf{Aspect} & \textbf{DV-QKD} & \textbf{CV-QKD} \\
\midrule
Information carrier & Single photons & Coherent states \\
Detection & Single-photon detectors & Homodyne/Heterodyne \\
Modulation & Discrete (2/4 states) & Continuous (Gaussian) \\
Detector technology & APD/SNSPD & Standard photodiodes \\
Key rate (short range) & Lower & Higher \\
Maximum distance & $\sim$400 km & $\sim$200 km \\
Telecom compatibility & Limited & High \\
\bottomrule
\end{tabular}
\end{table}

\subsection{Gaussian Modulation Protocol}

The GG02 protocol uses Gaussian-modulated coherent states:

\begin{equation}
\ket{\alpha} = e^{-|\alpha|^2/2} \sum_{n=0}^{\infty} \frac{\alpha^n}{\sqrt{n!}} \ket{n}
\label{eq:coherent_state}
\end{equation}

where $\alpha = x + ip$ with $x, p$ drawn from Gaussian distributions with variance $V_A$.

\subsection{Security Proofs}

Leverrier (2015) \cite{leverrier2015cv_security} established composable security for CV-QKD with coherent states, proving security against general attacks in the asymptotic limit. Key developments include:

\begin{itemize}
    \item \textbf{Collective attacks:} Security proven for arbitrary attack strategies with i.i.d. assumption
    \item \textbf{Finite-key analysis:} Composable security with practical key lengths
    \item \textbf{Trusted noise model:} Excess noise bounded by device characterization
\end{itemize}

\subsection{Relationship to Nguyen et al. (2021)}

Nguyen et al.'s QPSK-based approach bridges DV and CV paradigms:

\begin{itemize}
    \item Uses \textbf{discrete modulation} (4 phase states) like DV-QKD
    \item Employs \textbf{coherent detection} (heterodyne) like CV-QKD
    \item Security analysis based on \textbf{BB84 mapping}
\end{itemize}

This hybrid approach offers implementation simplicity while maintaining BB84-equivalent security.

% ============================================================================
% SECTION 3.4: DECOY STATE METHOD
% ============================================================================

\section{Decoy State Method}
\label{sec:decoy_state}

\subsection{Addressing Practical Source Limitations}

Practical QKD implementations use weak coherent pulses (WCP) instead of true single photons, introducing vulnerabilities to photon-number-splitting (PNS) attacks. The decoy state method, proposed by Hwang (2003) and refined by Lo, Ma, and Chen (2005), addresses this limitation.

\subsection{Principle}

Alice randomly varies the mean photon number $\mu$ of transmitted pulses:

\begin{itemize}
    \item \textbf{Signal state:} $\mu_s \approx 0.5$ (key generation)
    \item \textbf{Decoy state:} $\mu_d \approx 0.1$ (parameter estimation)
    \item \textbf{Vacuum state:} $\mu_v = 0$ (dark count estimation)
\end{itemize}

\subsection{Security Enhancement}

The decoy state method enables tight bounds on single-photon contribution:

\begin{equation}
Y_1^L \leq Y_1 \leq Y_1^U
\label{eq:decoy_bounds}
\end{equation}

where $Y_1$ is the single-photon yield, enabling security equivalent to ideal single-photon sources.

\subsection{Implementation in Satellite QKD}

The Micius satellite \cite{liao2017micius_qkd} demonstrated decoy-state BB84 for satellite QKD:
\begin{itemize}
    \item Three-intensity protocol ($\mu$, $\nu$, vacuum)
    \item QBER $\sim 1.1\%$ achieved
    \item Key rate up to 40.2 kbps at 530 km distance
\end{itemize}

% ============================================================================
% SECTION 3.5: REVIEW PAPERS
% ============================================================================

\section{Landmark Review Papers}
\label{sec:reviews}

\subsection{Gisin et al. (2002)}

``Quantum Cryptography'' in Reviews of Modern Physics \cite{gisin2002qkd_review} provided the first comprehensive review of QKD, covering:
\begin{itemize}
    \item Theoretical foundations and security proofs
    \item Experimental implementations
    \item Practical considerations and limitations
\end{itemize}

\textbf{Citations:} $>$5000 (foundational reference for the field)

\subsection{Scarani et al. (2009)}

``The Security of Practical Quantum Key Distribution'' \cite{scarani2009security} in Reviews of Modern Physics established the framework for analyzing practical QKD security:

\textbf{Key Contributions:}
\begin{itemize}
    \item Rigorous treatment of practical device imperfections
    \item Comprehensive attack classification
    \item Security parameter optimization
\end{itemize}

\textbf{Relevance to Nguyen et al.:} Provides the security framework for QBER analysis and threshold determination.

\subsection{Pirandola et al. (2020)}

``Advances in Quantum Cryptography'' \cite{pirandola2020advances_review} in Advances in Optics and Photonics (225 pages) represents the most comprehensive recent review:

\textbf{Coverage:}
\begin{itemize}
    \item DV-QKD and CV-QKD protocols
    \item Satellite and free-space implementations
    \item Quantum networks and repeaters
    \item Post-quantum considerations
\end{itemize}

\subsection{Xu et al. (2020)}

``Secure Quantum Key Distribution with Realistic Devices'' \cite{xu2020realistic_devices} in Reviews of Modern Physics addresses practical security:

\begin{itemize}
    \item Device imperfection modeling
    \item Side-channel attacks and countermeasures
    \item Measurement-device-independent QKD
    \item Twin-field QKD for extended range
\end{itemize}

% ============================================================================
% SECTION 3.6: TIMELINE
% ============================================================================

\section{Historical Timeline}
\label{sec:timeline}

\begin{table}[h]
\centering
\caption{QKD Development Timeline}
\label{tab:qkd_timeline}
\small
\begin{tabular}{cll}
\toprule
\textbf{Year} & \textbf{Milestone} & \textbf{Reference} \\
\midrule
1984 & BB84 protocol proposed & Bennett \& Brassard \\
1991 & E91 entanglement protocol & Ekert \\
1992 & First experimental BB84 (32 cm) & Bennett et al. \\
2002 & Comprehensive QKD review & Gisin et al. \\
2003 & CV-QKD with coherent states & Grosshans et al. \\
2005 & Decoy state method & Lo, Ma, Chen \\
2007 & 200 km fiber QKD & Schmitt-Manderbach \\
2009 & Practical security framework & Scarani et al. \\
2012 & Finite-key analysis & Tomamichel et al. \\
2015 & CV-QKD composable security & Leverrier \\
2016 & Micius satellite launch & USTC/CAS \\
2017 & First satellite QKD & Liao et al. \\
2020 & MDI-QKD over 500 km & Chen et al. \\
2021 & 4600 km integrated network & Chen et al. \\
\bottomrule
\end{tabular}
\end{table}

% ============================================================================
% SECTION 3.7: SUMMARY
% ============================================================================

\section{Chapter Summary}
\label{sec:ch3_summary}

This chapter established the theoretical foundations underlying Nguyen et al. (2021):

\begin{enumerate}
    \item \textbf{BB84 Protocol:} Provides the four-state structure mapped to QPSK phases
    \item \textbf{CV-QKD:} Introduces coherent detection applicable to heterodyne receivers
    \item \textbf{Decoy States:} Addresses practical source limitations in satellite implementations
    \item \textbf{Security Framework:} Establishes QBER thresholds and analysis methodology
\end{enumerate}

\textbf{Key Insight:} Nguyen et al.'s QPSK-DT/HD approach synthesizes concepts from multiple foundational works, creating a practical system that leverages coherent detection advantages while maintaining BB84-equivalent security structure.
