% ============================================================================
% CHAPTER 7: COMPARATIVE STUDY
% ============================================================================

\chapter{Comparative Study}
\label{chap:comparative}

\begin{quote}
\textit{This chapter positions the work by Nguyen et al. (2021) within the broader satellite QKD literature, comparing with experimental demonstrations, alternative approaches, and identifying the unique contributions.}
\end{quote}

% ============================================================================
% SECTION 7.1: COMPARISON WITH MICIUS
% ============================================================================

\section{Comparison with Micius Satellite QKD}
\label{sec:micius_comparison}

\subsection{System Parameters Comparison}

\begin{table}[h]
\centering
\caption{Nguyen et al. (2021) vs. Micius Satellite QKD}
\label{tab:micius_comparison}
\small
\begin{tabular}{lcc}
\toprule
\textbf{Parameter} & \textbf{Nguyen 2021 (Design)} & \textbf{Micius (Liao 2017)} \\
\midrule
\multicolumn{3}{l}{\textit{System Configuration}} \\
Satellite altitude & 600 km & 500 km \\
Protocol & QPSK-based (CV-like) & BB84 + Decoy states \\
Wavelength & 1550 nm & 850 nm \\
Modulation & Phase (QPSK) & Polarization \\
Detection & Heterodyne (APD) & Single-photon (Si-APD) \\
\midrule
\multicolumn{3}{l}{\textit{Performance}} \\
Design bit rate & 10 Gbps & -- \\
Achieved key rate & -- (simulation) & 1.1 kbps (at 1200 km) \\
QBER target & $< 10^{-3}$ & 1.1\% achieved \\
Channel loss & $\sim$260 dB (compensated) & 41.5 dB \\
\midrule
\multicolumn{3}{l}{\textit{Error Handling}} \\
Method & ARQ retransmission & FEC (LDPC) \\
Complexity & Low & High \\
Adaptability & Channel-adaptive & Fixed rate \\
\midrule
\multicolumn{3}{l}{\textit{Validation}} \\
Type & Numerical simulation & Experimental \\
Status & Proposed design & Demonstrated \\
\bottomrule
\end{tabular}
\end{table}

\subsection{Key Differences}

\textbf{1. Detection Approach:}
\begin{itemize}
    \item \textbf{Micius:} Single-photon detection with Si-APD at 850 nm
    \item \textbf{Nguyen:} Coherent (heterodyne) detection with InGaAs-APD at 1550 nm
\end{itemize}

\textbf{2. Protocol Type:}
\begin{itemize}
    \item \textbf{Micius:} Discrete-variable (DV-QKD) with decoy states
    \item \textbf{Nguyen:} Continuous-variable-like with QPSK modulation
\end{itemize}

\textbf{3. Wavelength Choice:}
\begin{itemize}
    \item \textbf{850 nm:} Better single-photon detector efficiency, higher scattering
    \item \textbf{1550 nm:} Telecom-compatible, lower atmospheric absorption
\end{itemize}

% ============================================================================
% SECTION 7.2: COMPARISON WITH CV-QKD
% ============================================================================

\section{Comparison with CV-QKD Approaches}
\label{sec:cvqkd_comparison}

\subsection{Gaussian CV-QKD vs. QPSK-Based}

\begin{table}[h]
\centering
\caption{Gaussian CV-QKD vs. QPSK-Based QKD}
\label{tab:cvqkd_comparison}
\begin{tabular}{lcc}
\toprule
\textbf{Aspect} & \textbf{Gaussian CV-QKD} & \textbf{QPSK-Based (Nguyen)} \\
\midrule
Modulation & Gaussian amplitude/phase & Discrete 4-phase \\
Alphabet & Continuous & Discrete (4 symbols) \\
Security proof & Well-established & Based on BB84 mapping \\
Detection & Homodyne/Heterodyne & Heterodyne with DT \\
Key rate (short dist.) & Higher & Moderate \\
Key rate (long dist.) & Lower & -- \\
Implementation & More complex & Simpler \\
Satellite feasibility & Under study & Proposed \\
\bottomrule
\end{tabular}
\end{table}

\subsection{Dequal et al. (2021) Feasibility Study}

Recent work on satellite CV-QKD feasibility \cite{dequal2021cv_qkd_satellite}:
\begin{itemize}
    \item Analyzed Gaussian-modulated CV-QKD for satellite
    \item Found positive key rate possible under certain conditions
    \item Highlighted excess noise as critical challenge
\end{itemize}

\textbf{Nguyen's approach differs by:}
\begin{itemize}
    \item Using discrete QPSK instead of Gaussian modulation
    \item Adding retransmission for reliability (not considered in Dequal)
    \item Focus on reliability metrics (KLR) in addition to key rate
\end{itemize}

% ============================================================================
% SECTION 7.3: COMPARISON WITH PTIT PRIOR WORK
% ============================================================================

\section{Evolution of PTIT Research}
\label{sec:ptit_evolution}

\subsection{Research Timeline}

\begin{table}[h]
\centering
\caption{PTIT Research Evolution in Satellite QKD}
\label{tab:ptit_evolution}
\small
\begin{tabular}{clll}
\toprule
\textbf{Year} & \textbf{Authors} & \textbf{Key Contribution} & \textbf{Advancement} \\
\midrule
2018 & Trinh et al. & DT/DD for QKD & First DT analysis \\
2019 & Vu et al. & QPSK + HD receiver & Added heterodyne \\
\textbf{2021} & \textbf{Nguyen et al.} & \textbf{+ Retransmission + Markov} & \textbf{Link layer reliability} \\
2023 & Nguyen et al. & CV-QKD optimization & Extended to CV-QKD \\
2023 & Phan et al. & CDMA-based CV-QKD & Multi-user extension \\
\bottomrule
\end{tabular}
\end{table}

\subsection{Incremental Contributions}

\textbf{Trinh et al. (2018) \cite{trinh2018dualthreshold_ptit}:}
\begin{itemize}
    \item Introduced dual-threshold detection for QKD
    \item Direct detection (DD) receiver
    \item Security analysis framework
\end{itemize}

\textbf{Nguyen et al. (2021) --- Current paper:}
\begin{itemize}
    \item Upgraded to heterodyne detection (20 dB improvement)
    \item Added key retransmission scheme (new contribution)
    \item Developed 3-D Markov chain model (analytical innovation)
    \item Comprehensive atmospheric channel model
\end{itemize}

% ============================================================================
% SECTION 7.4: MODULATION SCHEME COMPARISON
% ============================================================================

\section{Modulation Scheme Comparison}
\label{sec:modulation_comparison}

\begin{table}[h]
\centering
\caption{Comprehensive Modulation Scheme Comparison}
\label{tab:modulation_comparison}
\begin{tabular}{lccccc}
\toprule
\textbf{Scheme} & \textbf{States} & \textbf{Detection} & \textbf{Sensitivity} & \textbf{Complexity} & \textbf{Ref.} \\
\midrule
Polarization BB84 & 4 & SPD & High & Low & \cite{liao2017micius_qkd} \\
SIM/BPSK-DT & 2 & DD & Medium & Medium & \cite{trinh2018dualthreshold_ptit} \\
QPSK-DT/DD & 4 & DD & High & Medium & Vu 2019 \\
\textbf{QPSK-DT/HD} & \textbf{4} & \textbf{HD} & \textbf{Very High} & \textbf{Medium} & \textbf{Nguyen 2021} \\
Gaussian CV & $\infty$ & Homo/Hetero & High & High & \cite{grosshans2003cv_qkd} \\
\bottomrule
\end{tabular}
\end{table}

% ============================================================================
% SECTION 7.5: ERROR HANDLING COMPARISON
% ============================================================================

\section{Error Handling Approach Comparison}
\label{sec:error_comparison}

\begin{table}[h]
\centering
\caption{Error Handling Approaches in QKD}
\label{tab:error_handling}
\begin{tabular}{lcccc}
\toprule
\textbf{Method} & \textbf{Overhead} & \textbf{Latency} & \textbf{Adaptability} & \textbf{Used In} \\
\midrule
CASCADE & High & High & Low & Traditional QKD \\
LDPC & Medium & Medium & Low & Micius \\
Polar codes & Medium & Low & Low & Research \\
\textbf{ARQ (Nguyen)} & \textbf{Low} & \textbf{Variable} & \textbf{High} & \textbf{Proposed} \\
Hybrid FEC+ARQ & Medium & Variable & Medium & Future work \\
\bottomrule
\end{tabular}
\end{table}

% ============================================================================
% SECTION 7.6: UNIQUE CONTRIBUTIONS
% ============================================================================

\section{Unique Contributions of Nguyen et al. (2021)}
\label{sec:unique_contributions}

\subsection{What's New}

\begin{enumerate}
    \item \textbf{First ARQ-based reliability scheme for satellite QKD}
    \begin{itemize}
        \item Novel application of classical ARQ to quantum system
        \item Alternative to computationally expensive FEC
    \end{itemize}

    \item \textbf{3-D Markov chain model for KLR analysis}
    \begin{itemize}
        \item New analytical framework
        \item Enables performance prediction without extensive simulation
    \end{itemize}

    \item \textbf{Comprehensive integration}
    \begin{itemize}
        \item QPSK modulation + DT/HD detection + ARQ
        \item Physical layer and link layer co-design
    \end{itemize}

    \item \textbf{Practical parameter optimization}
    \begin{itemize}
        \item Clear guidelines for DT coefficient selection
        \item Optimal retransmission count determination
    \end{itemize}
\end{enumerate}

\subsection{Research Gaps Addressed}

\begin{table}[h]
\centering
\caption{Research Gaps Addressed by Nguyen et al. (2021)}
\label{tab:gaps_addressed}
\begin{tabular}{lcc}
\toprule
\textbf{Gap in Literature} & \textbf{Addressed?} & \textbf{How} \\
\midrule
Reliability improvement without FEC & \cmark & ARQ scheme \\
Link layer analysis for satellite QKD & \cmark & 3-D Markov model \\
QPSK + heterodyne for QKD & \cmark & DT/HD receiver \\
KLR metric for QKD & \cmark & Analytical derivation \\
Vietnamese satellite QKD research & \cmark & PTIT contribution \\
\bottomrule
\end{tabular}
\end{table}
