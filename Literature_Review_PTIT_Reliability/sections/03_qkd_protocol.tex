% ============================================================================
% CHAPTER 3: QKD PROTOCOL DESIGN
% ============================================================================

\chapter{QKD Protocol Design}
\label{chap:qkd_protocol}

\begin{quote}
\textit{This chapter examines the QPSK-based QKD protocol proposed by Nguyen et al. (2021), its relationship to the foundational BB84 protocol, and the dual-threshold/heterodyne detection scheme that enables improved performance.}
\end{quote}

% ============================================================================
% SECTION 3.1: FOUNDATIONAL CONTEXT
% ============================================================================

\section{Foundational Context: BB84 Protocol}
\label{sec:bb84}

\subsection{Original BB84 Protocol}

The BB84 protocol, proposed by Bennett and Brassard in 1984 \cite{bennett1984bb84}, established the foundation for QKD using four non-orthogonal quantum states:

\textbf{Two conjugate bases:}
\begin{itemize}
    \item Rectilinear basis ($+$): $\ket{H}$ (0), $\ket{V}$ (1)
    \item Diagonal basis ($\times$): $\ket{+45^\circ}$ (0), $\ket{-45^\circ}$ (1)
\end{itemize}

\textbf{Security foundation:}
\begin{enumerate}
    \item No-cloning theorem: Quantum states cannot be perfectly copied
    \item Measurement disturbance: Any measurement on unknown quantum state causes detectable change
    \item Basis mismatch: When Alice and Bob use different bases, result is random
\end{enumerate}

\subsection{Mapping to QPSK}

Nguyen et al. (2021) maps the BB84 four-state structure to QPSK phase states:

\begin{table}[h]
\centering
\caption{BB84 to QPSK State Mapping}
\label{tab:bb84_qpsk_mapping}
\begin{tabular}{cccc}
\toprule
\textbf{BB84 State} & \textbf{Polarization} & \textbf{QPSK Phase ($\phi_A$)} & \textbf{Bit} \\
\midrule
$\ket{H}$ & Horizontal & $\pi/4$ & 0 \\
$\ket{V}$ & Vertical & $5\pi/4$ & 1 \\
$\ket{+45^\circ}$ & Diagonal & $-\pi/4$ & 0 \\
$\ket{-45^\circ}$ & Anti-diagonal & $3\pi/4$ & 1 \\
\bottomrule
\end{tabular}
\end{table}

% ============================================================================
% SECTION 3.2: QPSK PROTOCOL
% ============================================================================

\section{QPSK-Based QKD Protocol}
\label{sec:qpsk_protocol}

\subsection{Protocol Overview}

The QPSK-based protocol operates in four steps, analogous to BB84:

\textbf{Step 1: Key Encoding (Alice)}
\begin{itemize}
    \item Alice randomly selects base $A_1$ or $A_2$
    \item For each binary bit, encodes into phase state $\phi_A$
    \item Phase formed by MZM: $\phi_A = (\phi_1 + \phi_2)/2$
\end{itemize}

\textbf{Step 2: Transmission and Detection (Bob)}
\begin{itemize}
    \item Bob randomly selects base $B_1$ ($\phi_B = \pi/4$) or $B_2$ ($\phi_B = -\pi/4$)
    \item Heterodyne detection produces current proportional to $\cos(\phi_A - \phi_B)$
    \item Dual-threshold decision: output $\in \{0, X, 1\}$
\end{itemize}

\textbf{Step 3: Sifting}
\begin{itemize}
    \item Bob announces times when he detected ``0'' or ``1'' (not the values)
    \item Alice discards bits where Bob detected ``X''
    \item Remaining bits form \textbf{sifted key}
\end{itemize}

\textbf{Step 4: Error Correction}
\begin{itemize}
    \item Instead of FEC, use \textbf{key retransmission scheme}
    \item Failed sequences are retransmitted up to $M$ times
\end{itemize}

\subsection{Complete Phase Encoding Table}

\begin{table}[h]
\centering
\caption{Complete QPSK-Based QKD Encoding}
\label{tab:encoding_table}
\small
\begin{tabular}{ccccc|ccc|cc}
\toprule
\textbf{Alice} & \textbf{Bit} & $\phi_1$ & $\phi_2$ & $\phi_A$ & \textbf{Bob} & $\phi_B$ & $\phi_A - \phi_B$ & \textbf{I} & \textbf{Result} \\
\midrule
$A_1$ & 0 & 0 & $\pi/2$ & $\pi/4$ & $B_1$ & $\pi/4$ & 0 & $I_0$ & 0 \\
$A_1$ & 0 & 0 & $\pi/2$ & $\pi/4$ & $B_2$ & $-\pi/4$ & $\pi/2$ & 0 & X \\
$A_1$ & 1 & $\pi$ & $3\pi/2$ & $5\pi/4$ & $B_1$ & $\pi/4$ & $\pi$ & $I_1$ & 1 \\
$A_1$ & 1 & $\pi$ & $3\pi/2$ & $5\pi/4$ & $B_2$ & $-\pi/4$ & $-\pi/2$ & 0 & X \\
\midrule
$A_2$ & 0 & 0 & $-\pi/2$ & $-\pi/4$ & $B_1$ & $\pi/4$ & $-\pi/2$ & 0 & X \\
$A_2$ & 0 & 0 & $-\pi/2$ & $-\pi/4$ & $B_2$ & $-\pi/4$ & 0 & $I_0$ & 0 \\
$A_2$ & 1 & $\pi$ & $\pi/2$ & $3\pi/4$ & $B_1$ & $\pi/4$ & $\pi/2$ & 0 & X \\
$A_2$ & 1 & $\pi$ & $\pi/2$ & $3\pi/4$ & $B_2$ & $-\pi/4$ & $\pi$ & $I_1$ & 1 \\
\bottomrule
\end{tabular}
\end{table}

\textbf{Key observation:} When Alice uses $A_i$ and Bob uses $B_i$ (same basis index), the correct bit is detected. Otherwise, result is ``X'' (erasure).

% ============================================================================
% SECTION 3.3: TRANSMITTER DESIGN
% ============================================================================

\section{Transmitter Design}
\label{sec:transmitter}

\subsection{QPSK Modulator Architecture}

The transmitter uses dual Mach-Zehnder modulators (MZMs):

\textbf{Components:}
\begin{enumerate}
    \item \textbf{Key Generator:} Produces random bit sequence $d(t)$
    \item \textbf{Buffer:} Stores bit sequences for potential retransmission
    \item \textbf{Electronic Controller:} Generates control signals based on bit values
    \item \textbf{Random Base Module:} Selects between $A_1$ and $A_2$
    \item \textbf{CW Laser:} Provides optical carrier
    \item \textbf{Dual MZM:} Encodes phase information
\end{enumerate}

\subsection{Optical Signal Formation}

The transmitted optical field:
\begin{equation}
E_T = \sqrt{P_T G_T} \exp\left[-i(2\pi f_c t + \phi_A)\right]
\label{eq:transmitted_field}
\end{equation}

where:
\begin{itemize}
    \item $P_T$: Peak transmitted power
    \item $G_T$: Transmitter telescope gain (120 dB)
    \item $f_c$: Optical carrier frequency
    \item $\phi_A$: Alice's phase encoding
\end{itemize}

% ============================================================================
% SECTION 3.4: RECEIVER DESIGN
% ============================================================================

\section{Dual-Threshold/Heterodyne Detection Receiver}
\label{sec:receiver}

\subsection{Heterodyne Detection Principle}

The received signal is mixed with a local oscillator (LO):
\begin{equation}
E_{LO} = \sqrt{P_{LO}} \exp\left[-i(2\pi f_{LO} t)\right]
\label{eq:lo_field}
\end{equation}

After photodetection and filtering:
\begin{equation}
I_{dec} = \bar{g}\Re\sqrt{P_R P_{LO}} \cos(\phi_A - \phi_B) + n(t)
\label{eq:decoded_current}
\end{equation}

where:
\begin{itemize}
    \item $\bar{g}$: Avalanche multiplication factor
    \item $\Re$: APD responsivity
    \item $P_R$: Received optical power
    \item $P_{LO}$: Local oscillator power
    \item $n(t)$: Noise current
\end{itemize}

\subsection{Dual-Threshold Decision}

After low-pass filtering, the signal $i(t)$ takes one of three forms:

\begin{equation}
i(t) = \begin{cases}
i_0 = \bar{g}\Re\sqrt{P_R P_{LO}} + n(t) & \text{for bit ``0''} \\
0 + n(t) & \text{for basis mismatch} \\
i_1 = -\bar{g}\Re\sqrt{P_R P_{LO}} + n(t) & \text{for bit ``1''}
\end{cases}
\label{eq:signal_cases}
\end{equation}

\textbf{Threshold levels:}
\begin{align}
d_0 &= E[i_0] + \varsigma\sqrt{\sigma_n^2} \label{eq:threshold_d0} \\
d_1 &= E[i_1] - \varsigma\sqrt{\sigma_n^2} \label{eq:threshold_d1}
\end{align}

where $\varsigma$ is the \textbf{dual-threshold scale coefficient}.

\subsection{Noise Analysis}

Total noise variance:
\begin{equation}
\sigma_n^2 = 2q\bar{g}^{2+x}\left[\Re(P_R + P_{LO}) + I_d\right]\Delta f + \frac{4k_B T}{R_L}\Delta f
\label{eq:noise_variance}
\end{equation}

\textbf{Noise components:}
\begin{itemize}
    \item Shot noise: $2q\bar{g}^{2+x}\Re(P_R + P_{LO})\Delta f$
    \item Dark current noise: $2q\bar{g}^{2+x}I_d\Delta f$
    \item Thermal noise: $\frac{4k_B T}{R_L}\Delta f$
\end{itemize}

where $\Delta f = R_b/2$ is the receiver bandwidth.

\subsection{Advantage of Heterodyne Detection}

\begin{table}[h]
\centering
\caption{Comparison of Detection Schemes}
\label{tab:detection_comparison}
\begin{tabular}{lccc}
\toprule
\textbf{Feature} & \textbf{Direct Detection} & \textbf{Homodyne} & \textbf{Heterodyne} \\
\midrule
Sensitivity & Low & High & High \\
LO required & No & Yes & Yes \\
Phase locking & N/A & Critical & Less critical \\
Bandwidth & Signal BW & Signal BW & 2$\times$ Signal BW \\
Quadratures & Intensity only & One & Both \\
Implementation & Simple & Moderate & Moderate \\
\bottomrule
\end{tabular}
\end{table}

% ============================================================================
% SECTION 3.5: QBER ANALYSIS
% ============================================================================

\section{Quantum Bit Error Rate Analysis}
\label{sec:qber_analysis}

\subsection{QBER Definition}

Following BB84:
\begin{equation}
\text{QBER} = \frac{P_{error}}{P_{sift}}
\label{eq:qber_definition}
\end{equation}

where:
\begin{align}
P_{error} &= P_{A,B}(0,1) + P_{A,B}(1,0) \label{eq:perror} \\
P_{sift} &= P_{A,B}(0,0) + P_{A,B}(0,1) + P_{A,B}(1,0) + P_{A,B}(1,1) \label{eq:psift}
\end{align}

\subsection{Joint Probability Calculation}

For joint probability $P_{A,B}(a,b)$ that Alice sends ``$a$'' and Bob detects ``$b$'':

\begin{equation}
P_{A,B}(a,b) = P_A(a) \cdot P_{B|A}(b|a)
\label{eq:joint_prob}
\end{equation}

with $P_A(a) = 1/2$ (equiprobable bits).

Averaged over fading channel:
\begin{align}
P_{A,B}(a,0) &= \frac{1}{2}\int_0^\infty Q\left(\frac{d_0 - I_a}{\sigma_n}\right) f_{h_f}(h_f) dh_f \label{eq:pab_0} \\
P_{A,B}(a,1) &= \frac{1}{2}\int_0^\infty Q\left(\frac{I_a - d_1}{\sigma_n}\right) f_{h_f}(h_f) dh_f \label{eq:pab_1}
\end{align}

where $Q(\cdot)$ is the Gaussian Q-function and $f_{h_f}(h_f)$ is the Gamma-Gamma PDF.

\subsection{DT Coefficient Optimization}

The dual-threshold coefficient $\varsigma$ controls the trade-off between QBER and $P_{sift}$:

\begin{itemize}
    \item \textbf{Large $\varsigma$:} Wider erasure region $\rightarrow$ Lower QBER, Lower $P_{sift}$
    \item \textbf{Small $\varsigma$:} Narrower erasure region $\rightarrow$ Higher QBER, Higher $P_{sift}$
\end{itemize}

\textbf{Optimal ranges (from numerical results):}
\begin{itemize}
    \item Weak turbulence ($C_n^2(0) = 5\times10^{-15}$): $0.7 \leq \varsigma \leq 2.4$
    \item Strong turbulence ($C_n^2(0) = 7\times10^{-12}$): $1.4 \leq \varsigma \leq 2.8$
\end{itemize}

These ranges satisfy: QBER $\leq 10^{-3}$ and $P_{sift} \geq 10^{-2}$.
