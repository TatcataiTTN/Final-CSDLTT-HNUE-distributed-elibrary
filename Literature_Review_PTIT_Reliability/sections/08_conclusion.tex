% ============================================================================
% CHAPTER 8: CONCLUSION
% ============================================================================

\chapter{Conclusion}
\label{chap:conclusion}

\begin{quote}
\textit{This chapter summarizes the key findings from the literature review, identifies remaining research gaps, and outlines future research directions for satellite-based QKD reliability improvement.}
\end{quote}

% ============================================================================
% SECTION 8.1: KEY FINDINGS
% ============================================================================

\section{Key Findings}
\label{sec:key_findings}

\subsection{Paper Contributions Summary}

Nguyen et al. (2021) makes four main contributions to satellite-based QKD:

\begin{enumerate}
    \item \textbf{QPSK-Based QKD Protocol with DT/HD Detection}
    \begin{itemize}
        \item Maps BB84 four-state structure to QPSK phase states
        \item Dual-threshold detection reduces QBER while maintaining $P_{sift}$
        \item Heterodyne detection provides 20 dB sensitivity improvement
        \item Achieves QBER $< 10^{-3}$ with $P_T = 25$ dBm
    \end{itemize}

    \item \textbf{Key Retransmission Scheme}
    \begin{itemize}
        \item Novel ARQ-based approach for QKD reliability
        \item Simple implementation without FEC computational overhead
        \item Achieves $>$1000$\times$ KLR improvement with $M = 4$ retransmissions
        \item Adapts naturally to varying channel conditions
    \end{itemize}

    \item \textbf{3-D Markov Chain Analytical Framework}
    \begin{itemize}
        \item Three-dimensional state space: (buffer, channel, retransmission)
        \item Enables closed-form KLR calculation
        \item Provides insights for system optimization
    \end{itemize}

    \item \textbf{Comprehensive Performance Analysis}
    \begin{itemize}
        \item Gamma-Gamma turbulence model with Hufnagel-Valley profile
        \item Weak and strong turbulence conditions evaluated
        \item Weather impact quantified
        \item Security against unauthorized receiver attack analyzed
    \end{itemize}
\end{enumerate}

\subsection{Quantitative Results Summary}

\begin{table}[h]
\centering
\caption{Summary of Quantitative Results}
\label{tab:results_summary}
\begin{tabular}{lc}
\toprule
\textbf{Metric} & \textbf{Result} \\
\midrule
Power improvement (vs. SIM/BPSK) & 20 dB \\
KLR improvement (with $M = 4$) & $>$1000$\times$ \\
Optimal DT coefficient (weak turb.) & 0.7 -- 2.4 \\
Optimal DT coefficient (strong turb.) & 1.4 -- 2.8 \\
Security distance (Eve-Bob) & $>$30 m \\
Optimal retransmission count & $M = 4$ \\
\bottomrule
\end{tabular}
\end{table}

% ============================================================================
% SECTION 8.2: STRENGTHS AND LIMITATIONS
% ============================================================================

\section{Strengths and Limitations}
\label{sec:strengths_limitations}

\subsection{Strengths}

\begin{enumerate}
    \item \textbf{Novel Integration:} First work combining QPSK, DT/HD, and ARQ for satellite QKD
    \item \textbf{Practical Focus:} Realistic system parameters based on LEO satellite
    \item \textbf{Analytical Rigor:} Mathematical framework enables performance prediction
    \item \textbf{Significant Improvement:} Quantifiable gains in power and reliability
    \item \textbf{Vietnamese Contribution:} Advances regional research capability
\end{enumerate}

\subsection{Limitations}

\begin{enumerate}
    \item \textbf{Simulation Only:} No experimental validation
    \item \textbf{Idealized Pointing:} Perfect beam tracking assumed
    \item \textbf{Asymptotic Security:} Finite-key effects not analyzed
    \item \textbf{Single Link:} No constellation or handover consideration
    \item \textbf{Simplified Eavesdropper:} Only URA scenario analyzed
\end{enumerate}

% ============================================================================
% SECTION 8.3: RESEARCH GAPS
% ============================================================================

\section{Remaining Research Gaps}
\label{sec:remaining_gaps}

\subsection{Theoretical Gaps}

\begin{enumerate}
    \item \textbf{Finite-Key Security Analysis}
    \begin{itemize}
        \item Security bounds for practical key lengths
        \item Minimum block size requirements
        \item Composable security proof
    \end{itemize}

    \item \textbf{Advanced Eavesdropper Models}
    \begin{itemize}
        \item Collective attacks
        \item Coherent attacks
        \item Side-channel vulnerabilities
    \end{itemize}
\end{enumerate}

\subsection{Practical Gaps}

\begin{enumerate}
    \item \textbf{Pointing and Tracking}
    \begin{itemize}
        \item Realistic pointing error models
        \item Acquisition and tracking protocols
        \item Impact on QBER and KLR
    \end{itemize}

    \item \textbf{Experimental Validation}
    \begin{itemize}
        \item Ground-based testbed demonstration
        \item Component characterization
        \item Satellite-analog experiments
    \end{itemize}
\end{enumerate}

\subsection{System-Level Gaps}

\begin{enumerate}
    \item \textbf{LEO Constellation Integration}
    \begin{itemize}
        \item Multi-satellite coverage
        \item Handover protocols
        \item Key management across satellites
    \end{itemize}

    \item \textbf{Hybrid Systems}
    \begin{itemize}
        \item Combined FSO/RF architecture
        \item Classical-quantum coexistence
        \item Network integration
    \end{itemize}
\end{enumerate}

% ============================================================================
% SECTION 8.4: FUTURE DIRECTIONS
% ============================================================================

\section{Future Research Directions}
\label{sec:future_directions}

\subsection{Near-Term (1-2 Years)}

\begin{enumerate}
    \item \textbf{Finite-Key Analysis Extension}
    \begin{itemize}
        \item Incorporate finite-size corrections into QBER analysis
        \item Determine minimum key length for target security
    \end{itemize}

    \item \textbf{Pointing Error Integration}
    \begin{itemize}
        \item Add realistic pointing jitter model
        \item Analyze combined pointing and turbulence effects
    \end{itemize}

    \item \textbf{Ground Testbed Development}
    \begin{itemize}
        \item Implement QPSK modulator and DT/HD receiver
        \item Validate with emulated satellite channel
    \end{itemize}
\end{enumerate}

\subsection{Medium-Term (3-5 Years)}

\begin{enumerate}
    \item \textbf{Machine Learning Integration}
    \begin{itemize}
        \item Channel prediction for adaptive parameters
        \item Optimal threshold selection
        \item Anomaly detection for security
    \end{itemize}

    \item \textbf{LEO Constellation Analysis}
    \begin{itemize}
        \item Multi-satellite coverage for Vietnam
        \item Handover optimization
        \item Key routing protocols
    \end{itemize}

    \item \textbf{Tropical Atmosphere Modeling}
    \begin{itemize}
        \item Vietnam-specific turbulence profiles
        \item Monsoon season characterization
        \item Ground station site selection
    \end{itemize}
\end{enumerate}

\subsection{Long-Term (5+ Years)}

\begin{enumerate}
    \item \textbf{Vietnamese Quantum Satellite Mission}
    \begin{itemize}
        \item Payload design based on PTIT research
        \item Ground station network
        \item International collaboration
    \end{itemize}

    \item \textbf{Regional Quantum Network}
    \begin{itemize}
        \item ASEAN quantum connectivity
        \item Cross-border secure communication
        \item Standards development
    \end{itemize}
\end{enumerate}

% ============================================================================
% SECTION 8.5: FINAL REMARKS
% ============================================================================

\section{Final Remarks}
\label{sec:final_remarks}

Nguyen et al. (2021) represents a significant contribution to satellite-based QKD research, particularly from the Vietnamese research community. The paper addresses a practical challenge---reliability improvement---through a novel combination of physical layer optimization (QPSK-DT/HD) and link layer mechanisms (ARQ retransmission).

The 20 dB power improvement and $>$1000$\times$ KLR reduction demonstrated in simulation suggest that the proposed approach could enable more practical satellite QKD systems. The analytical 3-D Markov chain model provides a valuable framework for system design and optimization.

While experimental validation remains necessary, the work establishes a foundation for future Vietnamese contributions to global quantum communication research. As satellite QKD moves toward operational deployment, the reliability techniques developed here may prove essential for practical system implementation.

\vspace{1cm}
\hrule
\vspace{0.5cm}
\textit{This literature review was prepared as part of the Master's program in Space \& Earth Observation at USTH, December 2025.}
