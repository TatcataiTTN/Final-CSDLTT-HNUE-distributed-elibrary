% ============================================================================
% CHAPTER 1: INTRODUCTION
% ============================================================================

\chapter{Introduction}
\label{chap:introduction}

\begin{quote}
\textit{This literature review provides a comprehensive analysis of reliability improvement techniques for satellite-based Quantum Key Distribution (QKD) systems, with particular focus on the retransmission scheme proposed by Nguyen et al. (2021) from the Posts and Telecommunications Institute of Technology (PTIT), Vietnam.}
\end{quote}

% ============================================================================
% SECTION 1.1: PAPER OVERVIEW
% ============================================================================

\section{Paper Under Review}
\label{sec:paper_overview}

\subsection{Bibliographic Information}

\begin{table}[h]
\centering
\caption{Paper Identification}
\label{tab:paper_id}
\begin{tabular}{ll}
\toprule
\textbf{Field} & \textbf{Information} \\
\midrule
Title & Reliability improvement of satellite-based quantum key \\
      & distribution systems using retransmission scheme \\
Authors & Nam D. Nguyen, Hang T. T. Phan, Hien T. T. Pham, \\
        & Vuong V. Mai, Ngoc T. Dang \\
Journal & Photonic Network Communications \\
Publisher & Springer \\
Year & 2021 \\
DOI & 10.1007/s11107-021-00934-y \\
Institution & Posts and Telecommunications Institute of Technology (PTIT) \\
Country & Vietnam \\
\bottomrule
\end{tabular}
\end{table}

\subsection{Abstract Summary}

The paper addresses the design and performance analysis of reliable satellite-based QKD over free-space optics (FSO) channels. The key contributions include:

\begin{enumerate}
    \item \textbf{QPSK-based QKD Protocol:} Optical quadrature phase-shift keying modulation adapted for quantum key distribution
    \item \textbf{Dual-Threshold/Heterodyne Detection (DT/HD):} Advanced receiver design that reduces QBER and improves sensitivity
    \item \textbf{Key Retransmission Scheme:} ARQ-based protocol at the link layer to enhance reliability
    \item \textbf{3-D Markov Chain Model:} Novel analytical framework for Key Loss Rate (KLR) analysis
\end{enumerate}

% ============================================================================
% SECTION 1.2: RESEARCH CONTEXT
% ============================================================================

\section{Research Context and Motivation}
\label{sec:research_context}

\subsection{The Need for Quantum-Secure Communication}

The development of quantum computing poses fundamental threats to classical cryptographic systems. Shor's algorithm, when implemented on a sufficiently powerful quantum computer, can efficiently factor large integers, thereby breaking RSA and elliptic curve cryptography---the foundations of modern secure communication \cite{scarani2009security}.

\textbf{Timeline of Quantum Computing Threat:}
\begin{itemize}
    \item \textbf{Current (2025):} NISQ-era quantum computers with $\sim$1000 qubits
    \item \textbf{Near-term (2030):} Potential for cryptographically-relevant quantum computers
    \item \textbf{Harvest now, decrypt later:} Adversaries may store encrypted data today for future decryption
\end{itemize}

Quantum Key Distribution (QKD) offers a solution with information-theoretic security---security guaranteed by the laws of physics rather than computational assumptions.

\subsection{Why Satellite-Based QKD?}

While fiber-based QKD has achieved commercial deployment, fundamental limitations restrict its range:

\begin{table}[h]
\centering
\caption{Comparison of QKD Transmission Media}
\label{tab:qkd_media}
\begin{tabular}{lcc}
\toprule
\textbf{Medium} & \textbf{Maximum Distance} & \textbf{Limitation} \\
\midrule
Optical Fiber & $\sim$400 km & Exponential attenuation ($\sim$0.2 dB/km) \\
Terrestrial FSO & $\sim$10 km & Atmospheric turbulence, weather \\
\textbf{Satellite FSO} & \textbf{$>$1000 km} & \textbf{Lower atmospheric path length} \\
\bottomrule
\end{tabular}
\end{table}

\textbf{Key advantages of satellite QKD:}
\begin{itemize}
    \item Free-space loss scales as $1/R^2$ (better than exponential fiber loss for long distances)
    \item Vacuum of space has negligible absorption
    \item Single satellite can serve multiple ground stations
    \item Global coverage possible with constellation
\end{itemize}

\subsection{The Reliability Challenge}

Despite its promise, satellite QKD faces significant reliability challenges:

\begin{enumerate}
    \item \textbf{Atmospheric Turbulence:} Random intensity fluctuations (scintillation) cause signal fading
    \item \textbf{Free-Space Path Loss:} $>$40 dB loss for LEO satellites at 600 km altitude
    \item \textbf{Weather Dependence:} Clouds, rain, and aerosols increase attenuation
    \item \textbf{Beam Spreading:} Diffraction causes power dilution at receiver
    \item \textbf{Pointing Errors:} Misalignment between satellite and ground station
    \item \textbf{Background Noise:} Solar radiation during daytime operation
\end{enumerate}

These factors lead to high Quantum Bit Error Rate (QBER) and potential key transmission failures, motivating the need for reliability improvement techniques.

% ============================================================================
% SECTION 1.3: VIETNAMESE RESEARCH CONTEXT
% ============================================================================

\section{Vietnamese Research Context}
\label{sec:vietnam_context}

\subsection{PTIT Research Group}

The Posts and Telecommunications Institute of Technology (PTIT) in Hanoi has established itself as a leading center for optical wireless communication research in Vietnam. Key achievements include:

\begin{itemize}
    \item \textbf{Dual-threshold detection analysis:} Trinh et al. (2018) \cite{trinh2018dualthreshold_ptit}
    \item \textbf{Reliability improvement schemes:} Nguyen et al. (2021) --- the paper under review
    \item \textbf{CV-QKD optimization:} Nguyen et al. (2023) \cite{nguyen2023cv_qkd_ptit}
    \item \textbf{International collaborations:} University of Aizu (Japan), KAIST (Korea)
\end{itemize}

\subsection{Regional Significance}

Vietnam's strategic location and growing technological capabilities position it well for quantum communication development:

\begin{itemize}
    \item Vietnam National Space Center (VNSC) under VAST
    \item USTH graduate programs in space technology
    \item Regional cooperation within ASEAN
    \item Tropical atmosphere conditions requiring specific modeling
\end{itemize}

% ============================================================================
% SECTION 1.4: REVIEW OBJECTIVES
% ============================================================================

\section{Literature Review Objectives}
\label{sec:objectives}

This literature review aims to:

\begin{enumerate}
    \item \textbf{Comprehensive Analysis:} Provide in-depth understanding of Nguyen et al. (2021) contributions

    \item \textbf{Theoretical Foundation:} Connect the paper to foundational QKD and FSO literature

    \item \textbf{Technical Deep-Dive:} Analyze the QPSK-based protocol, DT/HD detection, and retransmission scheme

    \item \textbf{Mathematical Framework:} Review the channel model, QBER derivation, and 3-D Markov chain analysis

    \item \textbf{Performance Evaluation:} Understand numerical results and their implications

    \item \textbf{Critical Assessment:} Identify strengths, limitations, and future research directions

    \item \textbf{Comparative Context:} Position the work within the broader satellite QKD literature
\end{enumerate}

% ============================================================================
% SECTION 1.5: REVIEW STRUCTURE
% ============================================================================

\section{Review Structure}
\label{sec:structure}

This literature review is organized as follows:

\textbf{Chapter 2: Detailed Paper Analysis} presents the complete technical analysis of Nguyen et al. (2021), including system architecture, key innovations, and main results.

\textbf{Chapter 3: QKD Protocol Design} examines the QPSK-based QKD protocol, its relationship to BB84, and the dual-threshold/heterodyne detection scheme.

\textbf{Chapter 4: Channel Model Analysis} provides detailed analysis of the FSO channel model including free-space loss, atmospheric attenuation, beam spreading, and Gamma-Gamma turbulence fading.

\textbf{Chapter 5: Retransmission Scheme} analyzes the key retransmission protocol, the 3-D Markov chain model, and Key Loss Rate derivation.

\textbf{Chapter 6: Performance Analysis} reviews numerical results for QBER, $P_{sift}$, and KLR under various conditions.

\textbf{Chapter 7: Comparative Study} positions the work within the broader literature and compares with alternative approaches.

\textbf{Chapter 8: Conclusion} synthesizes key findings and outlines future research directions.
