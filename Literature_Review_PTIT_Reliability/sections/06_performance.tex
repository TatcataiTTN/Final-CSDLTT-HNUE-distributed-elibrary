% ============================================================================
% CHAPTER 6: PERFORMANCE ANALYSIS
% ============================================================================

\chapter{Performance Analysis}
\label{chap:performance}

\begin{quote}
\textit{This chapter reviews the numerical results presented by Nguyen et al. (2021), including physical layer performance (QBER, $P_{sift}$), link layer performance (KLR), and security analysis against unauthorized receiver attacks.}
\end{quote}

% ============================================================================
% SECTION 6.1: PHYSICAL LAYER RESULTS
% ============================================================================

\section{Physical Layer Performance}
\label{sec:physical_performance}

\subsection{QBER and $P_{sift}$ vs. DT Coefficient}

The dual-threshold coefficient $\varsigma$ controls the trade-off between QBER and sifting probability.

\begin{table}[h]
\centering
\caption{Optimal DT Coefficient Ranges}
\label{tab:dt_ranges}
\begin{tabular}{lccc}
\toprule
\textbf{Condition} & \textbf{$\varsigma$ Range} & \textbf{QBER Target} & \textbf{$P_{sift}$ Target} \\
\midrule
Weak turbulence & $0.7 \leq \varsigma \leq 2.4$ & $\leq 10^{-3}$ & $\geq 10^{-2}$ \\
Strong turbulence & $1.4 \leq \varsigma \leq 2.8$ & $\leq 10^{-3}$ & $\geq 10^{-2}$ \\
\bottomrule
\end{tabular}
\end{table}

\textbf{Key observations:}
\begin{itemize}
    \item Increasing $\varsigma$ reduces both QBER and $P_{sift}$
    \item Wider erasure region ($|d_0 - d_1|$) rejects more uncertain detections
    \item Trade-off: Lower errors vs. lower key rate
\end{itemize}

\subsection{Transmitted Power Requirements}

Comparison of modulation/detection schemes for QBER $\leq 10^{-3}$ under weak turbulence:

\begin{table}[h]
\centering
\caption{Required Transmitted Power Comparison}
\label{tab:power_requirements}
\begin{tabular}{lcc}
\toprule
\textbf{Scheme} & \textbf{Min. $P_T$ (dBm)} & \textbf{Gain vs. SIM/BPSK} \\
\midrule
SIM/BPSK-DT & 45 & Baseline \\
QPSK-DT/DD & 35 & 10 dB \\
\textbf{QPSK-DT/HD} & \textbf{25} & \textbf{20 dB} \\
\bottomrule
\end{tabular}
\end{table}

\textbf{Significance:} 20 dB power reduction enables:
\begin{itemize}
    \item Smaller satellite transmitter
    \item Extended operational range
    \item Better margin for adverse conditions
\end{itemize}

\subsection{Weather Impact}

QBER performance under different attenuation coefficients (strong turbulence, $\varsigma = 1.4$):

\begin{table}[h]
\centering
\caption{Weather Impact on QBER}
\label{tab:weather_impact}
\begin{tabular}{lccc}
\toprule
\textbf{Condition} & \textbf{$\gamma$ (dB/km)} & \textbf{$P_T$ for QBER $\leq 10^{-3}$} & \textbf{Feasibility} \\
\midrule
Very clear & 0 -- 0.5 & 45 dBm & \cmark \\
Light rain/mist & 0.5 -- 1.53 & 45 dBm & \cmark \\
Haze/medium rain & 1.54 -- 2.68 & 50--55 dBm & Marginal \\
Heavy rain & $>$ 2.68 & $>$ 55 dBm & \xmark \\
\bottomrule
\end{tabular}
\end{table}

% ============================================================================
% SECTION 6.2: LINK LAYER RESULTS
% ============================================================================

\section{Link Layer Performance}
\label{sec:link_performance}

\subsection{KLR vs. Transmitted Power}

Under weak turbulence ($C_n^2(0) = 5 \times 10^{-15}$, $\varsigma = 0.7$):

\begin{table}[h]
\centering
\caption{KLR Performance with Retransmissions (Weak Turbulence)}
\label{tab:klr_weak}
\begin{tabular}{cccc}
\toprule
\textbf{Scheme} & \textbf{KLR at $P_T = 30$ dBm} & \textbf{KLR at $P_T = 34$ dBm} & \textbf{Improvement} \\
\midrule
No retransmission & $3 \times 10^{-2}$ & $3 \times 10^{-2}$ & Baseline \\
$M = 1$ & $10^{-3}$ & $10^{-4}$ & 10--100$\times$ \\
$M = 2$ & $10^{-4}$ & $10^{-5}$ & 100--1000$\times$ \\
$M = 4$ & $10^{-5}$ & $10^{-6}$ & 1000--10000$\times$ \\
$M = 7$ & $10^{-6}$ & $< 10^{-6}$ & $>$ 10000$\times$ \\
\bottomrule
\end{tabular}
\end{table}

\textbf{Key finding:} Without retransmission, KLR floors at $\sim 3\%$ even with high power.

\subsection{KLR vs. Transmitted Power (Strong Turbulence)}

Under strong turbulence ($C_n^2(0) = 7 \times 10^{-12}$, $\varsigma = 1.4$):

\textbf{Observations:}
\begin{itemize}
    \item Higher $P_T$ required for same KLR
    \item At KLR $= 10^{-6}$: 2 dB power gain when $M$ increases from 1 to 4
    \item Only 0.5 dB additional gain from $M = 4$ to $M = 7$
\end{itemize}

\textbf{Practical recommendation:} $M = 4$ offers best trade-off between reliability and latency.

\subsection{Diminishing Returns Analysis}

\begin{table}[h]
\centering
\caption{Power Gain vs. Retransmission Count at KLR $= 10^{-6}$}
\label{tab:diminishing_returns}
\begin{tabular}{ccc}
\toprule
\textbf{$M$ Increase} & \textbf{Power Gain (dB)} & \textbf{Latency Impact} \\
\midrule
1 $\rightarrow$ 2 & 1.0 & +1 slot max \\
2 $\rightarrow$ 3 & 0.7 & +1 slot max \\
3 $\rightarrow$ 4 & 0.5 & +1 slot max \\
4 $\rightarrow$ 5 & 0.3 & +1 slot max \\
5 $\rightarrow$ 7 & 0.2 & +2 slots max \\
\bottomrule
\end{tabular}
\end{table}

% ============================================================================
% SECTION 6.3: SECURITY ANALYSIS
% ============================================================================

\section{Security Analysis}
\label{sec:security_analysis}

\subsection{Unauthorized Receiver Attack (URA) Model}

\textbf{Scenario:}
\begin{itemize}
    \item Eve places unauthorized receiver near Bob (within beam footprint)
    \item Bob at beam center ($r = 0$)
    \item Eve at distance $D_{E-B}$ from Bob
\end{itemize}

\textbf{Eve's received power:}
\begin{equation}
P_{R,Eve} \propto h_l(D_{E-B}; D_{SG}) = A_0 \exp\left(-\frac{2D_{E-B}^2}{\omega_{Deq}^2}\right)
\label{eq:eve_power}
\end{equation}

\subsection{Eve's QBER vs. Distance}

As Eve moves away from Bob:
\begin{itemize}
    \item Received power decreases exponentially
    \item SNR degrades
    \item Eve's QBER increases
\end{itemize}

\begin{table}[h]
\centering
\caption{Eve's Performance vs. Distance from Bob}
\label{tab:eve_performance}
\begin{tabular}{lccc}
\toprule
\textbf{$D_{E-B}$ (m)} & \textbf{Eve's QBER (Weak)} & \textbf{Eve's QBER (Strong)} & \textbf{Security} \\
\midrule
0 & Same as Bob & Same as Bob & \xmark Compromised \\
10 & $\sim 10^{-3}$ & $\sim 10^{-2}$ & \xmark Marginal \\
20 & $\sim 10^{-2}$ & $\sim 10^{-2}$ & Marginal \\
\textbf{30} & $\mathbf{> 10^{-2}}$ & $\mathbf{> 10^{-2}}$ & \cmark \textbf{Secure} \\
50 & $\sim 10^{-1}$ & $\sim 10^{-1}$ & \cmark Secure \\
100 & $> 10^{-1}$ & $> 10^{-1}$ & \cmark Secure \\
\bottomrule
\end{tabular}
\end{table}

\subsection{Security Boundary}

\textbf{Requirement:} Eve's QBER $> 10^{-2}$ (cannot correct errors even with FEC)

\textbf{Result:} Minimum secure distance $D_{E-B} > 30$ m for both weak and strong turbulence.

\subsection{Eve's Trade-off}

If Eve increases her DT coefficient to reduce QBER:
\begin{itemize}
    \item Her $P_{sift}$ decreases correspondingly
    \item Less information obtained from Alice
    \item Cannot simultaneously achieve low QBER and high $P_{sift}$ at distance
\end{itemize}

% ============================================================================
% SECTION 6.4: SUMMARY
% ============================================================================

\section{Performance Summary}
\label{sec:performance_summary}

\begin{table}[h]
\centering
\caption{Overall Performance Summary}
\label{tab:performance_summary}
\begin{tabular}{lcc}
\toprule
\textbf{Metric} & \textbf{Conventional} & \textbf{Proposed (QPSK-DT/HD + ARQ)} \\
\midrule
Required $P_T$ (QBER $\leq 10^{-3}$) & 45 dBm & 25 dBm \\
KLR (without retx) & $3 \times 10^{-2}$ & -- \\
KLR (with $M = 4$) & -- & $< 10^{-4}$ \\
Power gain & Baseline & 20 dB \\
Reliability improvement & Baseline & $>$ 1000$\times$ \\
Security distance & -- & 30 m \\
\bottomrule
\end{tabular}
\end{table}
