% ============================================================================
% CHAPTER 11: CONCLUSION
% ============================================================================

\chapter{Conclusion}
\label{chap:conclusion}

\begin{quote}
\textit{This chapter synthesizes the key findings from this comprehensive literature review, summarizes the contributions of Nguyen et al. (2021), and provides final recommendations for future research.}
\end{quote}

% ============================================================================
% SECTION 11.1: LITERATURE REVIEW SUMMARY
% ============================================================================

\section{Literature Review Summary}
\label{sec:review_summary}

This literature review examined 85+ papers spanning four decades of quantum key distribution research, organized into four thematic parts:

\subsection{Part I: Introduction and Paper Analysis}

\begin{itemize}
    \item Established the context for satellite-based QKD research
    \item Provided detailed technical analysis of Nguyen et al. (2021)
    \item Identified system architecture, innovations, and key results
\end{itemize}

\subsection{Part II: Theoretical Foundations}

\begin{itemize}
    \item \textbf{Foundational Protocols:} BB84, E91, CV-QKD, and decoy states
    \item \textbf{Satellite Experiments:} Micius achievements and integrated networks
    \item \textbf{Key Insight:} Nguyen et al. builds upon established foundations while introducing novel reliability mechanisms
\end{itemize}

\subsection{Part III: Technical Aspects}

\begin{itemize}
    \item \textbf{Channel Models:} Gamma-Gamma turbulence, Hufnagel-Valley profile
    \item \textbf{Detection Schemes:} Heterodyne detection, dual-threshold approach
    \item \textbf{Error Handling:} CASCADE, LDPC, polar codes, and ARQ
    \item \textbf{Security:} Practical security frameworks and finite-key analysis
\end{itemize}

\subsection{Part IV: Analysis and Synthesis}

\begin{itemize}
    \item \textbf{Comparative Analysis:} Positioned Nguyen et al. within broader literature
    \item \textbf{Research Gaps:} Identified theoretical, practical, and system-level gaps
    \item \textbf{Future Directions:} Outlined near-term to long-term research roadmap
\end{itemize}

% ============================================================================
% SECTION 11.2: KEY FINDINGS
% ============================================================================

\section{Key Findings}
\label{sec:key_findings}

\subsection{Nguyen et al. (2021) Contributions}

The paper makes four significant contributions to satellite-based QKD:

\begin{table}[h]
\centering
\caption{Summary of Paper Contributions}
\label{tab:contributions_summary}
\begin{tabular}{lcc}
\toprule
\textbf{Contribution} & \textbf{Type} & \textbf{Impact} \\
\midrule
QPSK-based QKD with DT/HD & Physical Layer & 20 dB power improvement \\
Key retransmission scheme & Link Layer & $>$1000$\times$ KLR reduction \\
3-D Markov chain model & Analytical & Enables optimization \\
Comprehensive analysis & System & Practical guidelines \\
\bottomrule
\end{tabular}
\end{table}

\subsection{Quantitative Results}

\begin{table}[h]
\centering
\caption{Summary of Quantitative Results}
\label{tab:quantitative_summary}
\begin{tabular}{lc}
\toprule
\textbf{Metric} & \textbf{Result} \\
\midrule
Power improvement vs. SIM/BPSK & 20 dB \\
KLR improvement with $M = 4$ & $>$1000$\times$ \\
Optimal DT coefficient (weak turbulence) & 0.7 -- 2.4 \\
Optimal DT coefficient (strong turbulence) & 1.4 -- 2.8 \\
Security distance (Eve-Bob) & $>$30 m \\
Optimal retransmission count & $M = 4$ \\
\bottomrule
\end{tabular}
\end{table}

\subsection{Unique Position in Literature}

Nguyen et al. occupies a unique position by:

\begin{enumerate}
    \item Being the \textbf{first} to apply ARQ retransmission to satellite QKD
    \item Providing the \textbf{first} 3-D Markov chain model for link-layer QKD analysis
    \item Achieving \textbf{highest sensitivity} through DT/HD combination
    \item Demonstrating \textbf{cross-layer optimization} (physical + link layer)
\end{enumerate}

% ============================================================================
% SECTION 11.3: STRENGTHS AND LIMITATIONS
% ============================================================================

\section{Critical Assessment}
\label{sec:critical_assessment}

\subsection{Strengths}

\begin{enumerate}
    \item \textbf{Novel Integration:} First work combining QPSK, DT/HD, and ARQ for satellite QKD
    \item \textbf{Practical Focus:} Realistic system parameters based on LEO satellite configuration
    \item \textbf{Analytical Rigor:} Mathematical framework enables performance prediction without extensive simulation
    \item \textbf{Significant Improvement:} Quantifiable gains that could enable practical deployment
    \item \textbf{Vietnamese Contribution:} Advances regional research capability in quantum communications
\end{enumerate}

\subsection{Limitations}

\begin{enumerate}
    \item \textbf{Simulation Only:} No experimental validation of theoretical predictions
    \item \textbf{Idealized Pointing:} Perfect beam tracking assumed
    \item \textbf{Asymptotic Security:} Finite-key effects not analyzed
    \item \textbf{Single Link:} No constellation or handover consideration
    \item \textbf{Simplified Eavesdropper:} Only URA scenario analyzed
\end{enumerate}

% ============================================================================
% SECTION 11.4: RECOMMENDATIONS
% ============================================================================

\section{Recommendations}
\label{sec:recommendations}

\subsection{For Researchers}

\begin{enumerate}
    \item \textbf{Priority 1 - Finite-Key Analysis:}
    \begin{itemize}
        \item Extend security analysis to practical key lengths
        \item Determine minimum block sizes for target security levels
    \end{itemize}

    \item \textbf{Priority 2 - Experimental Validation:}
    \begin{itemize}
        \item Develop ground testbed for DT/HD receiver
        \item Validate ARQ protocol with emulated channel
    \end{itemize}

    \item \textbf{Priority 3 - Pointing Integration:}
    \begin{itemize}
        \item Add realistic pointing error models
        \item Analyze combined turbulence and pointing effects
    \end{itemize}
\end{enumerate}

\subsection{For Practitioners}

\begin{enumerate}
    \item Use $M = 4$ retransmissions as optimal starting point
    \item Select DT coefficient based on turbulence regime (0.7--2.8 range)
    \item Consider DT/HD approach for telecom-compatible implementations
    \item Plan for $>$30 m security perimeter around ground stations
\end{enumerate}

\subsection{For Policymakers}

\begin{enumerate}
    \item Support experimental validation of Vietnamese QKD research
    \item Consider satellite QKD in national quantum communication strategy
    \item Explore regional cooperation for ASEAN quantum network
    \item Invest in ground station infrastructure development
\end{enumerate}

% ============================================================================
% SECTION 11.5: FUTURE OUTLOOK
% ============================================================================

\section{Future Outlook}
\label{sec:outlook}

\subsection{Technology Trajectory}

Satellite QKD is progressing from experimental demonstrations toward operational deployment:

\begin{itemize}
    \item \textbf{2025-2026:} Eagle-1, QUBE-II, expanded Chinese constellation
    \item \textbf{2027:} Chinese MEO satellite, global service announcement
    \item \textbf{2030+:} Quantum internet backbone integration
\end{itemize}

\subsection{Vietnamese Opportunity}

Vietnam has opportunity to participate in this development through:

\begin{itemize}
    \item Continued theoretical research building on PTIT foundation
    \item Ground station development and characterization
    \item Regional collaboration with ASEAN partners
    \item Potential contribution to international missions
\end{itemize}

\subsection{Role of Nguyen et al. (2021)}

The work by Nguyen et al. provides:

\begin{itemize}
    \item Theoretical foundation for Vietnamese satellite QKD development
    \item Novel approaches (ARQ, 3-D Markov) applicable to broader community
    \item Framework for future experimental validation
    \item Basis for continued research advancement
\end{itemize}

% ============================================================================
% SECTION 11.6: FINAL REMARKS
% ============================================================================

\section{Final Remarks}
\label{sec:final_remarks}

Nguyen et al. (2021) represents a significant contribution to satellite-based QKD research, particularly from the Vietnamese research community. The paper addresses a practical challenge---reliability improvement---through a novel combination of physical layer optimization (QPSK-DT/HD) and link layer mechanisms (ARQ retransmission).

The 20 dB power improvement and $>$1000$\times$ KLR reduction demonstrated in simulation suggest that the proposed approach could enable more practical satellite QKD systems. The analytical 3-D Markov chain model provides a valuable framework for system design and optimization that has not been previously available in the literature.

While experimental validation remains necessary before practical deployment, the work establishes a solid foundation for future Vietnamese contributions to global quantum communication research. As satellite QKD moves toward operational deployment in the coming years, the reliability techniques developed here may prove essential for practical system implementation.

This literature review has situated Nguyen et al. within the broader context of 40 years of QKD research, identified its unique contributions, and outlined pathways for continued advancement. The field of satellite-based quantum communication holds great promise for enabling truly secure global communications, and Vietnamese researchers are positioned to contribute meaningfully to this important endeavor.

\vspace{1cm}
\begin{center}
\rule{0.5\textwidth}{0.5pt}
\end{center}
\vspace{0.5cm}

\begin{center}
\textit{This comprehensive literature review was prepared as part of the Master's program in Space \& Earth Observation at USTH, December 2025.}
\end{center}
