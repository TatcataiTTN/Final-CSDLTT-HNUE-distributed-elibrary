% ============================================================================
% CHAPTER 4: SATELLITE QKD EXPERIMENTS
% ============================================================================

\chapter{Satellite QKD Experiments}
\label{chap:satellite_experiments}

\begin{quote}
\textit{This chapter reviews the landmark experimental demonstrations of satellite-based QKD, from the Micius satellite missions to the integrated space-ground quantum network, providing the experimental context for Nguyen et al.'s theoretical contributions.}
\end{quote}

% ============================================================================
% SECTION 4.1: THE MICIUS SATELLITE
% ============================================================================

\section{The Micius Satellite (QUESS)}
\label{sec:micius}

\subsection{Mission Overview}

The Quantum Experiments at Space Scale (QUESS) mission, featuring the Micius satellite, represents humanity's first dedicated quantum science satellite. Named after the ancient Chinese philosopher Mozi (墨子), this \$100 million mission was launched on August 16, 2016 \cite{liao2017micius_qkd}.

\begin{table}[h]
\centering
\caption{Micius Satellite Specifications}
\label{tab:micius_specs}
\begin{tabular}{ll}
\toprule
\textbf{Parameter} & \textbf{Specification} \\
\midrule
Launch Date & August 16, 2016 \\
Launch Vehicle & Long March 2D \\
Orbit & Sun-synchronous LEO \\
Altitude & $\sim$500 km \\
Inclination & 97.4° \\
Mass & 631 kg \\
Design Life & 2 years (exceeded) \\
Operating Institution & CAS/USTC \\
Lead Scientist & Prof. Jian-Wei Pan \\
\bottomrule
\end{tabular}
\end{table}

\subsection{Scientific Payload}

The satellite carries three main experimental payloads:

\begin{enumerate}
    \item \textbf{QKD Transmitter:} Decoy-state BB84 source
    \begin{itemize}
        \item Wavelength: 850 nm
        \item Repetition rate: 100 MHz
        \item Decoy intensities: $\mu$, $\nu$, vacuum
    \end{itemize}

    \item \textbf{Entanglement Source:} Spontaneous parametric down-conversion
    \begin{itemize}
        \item Entangled photon pairs at 810 nm
        \item $>$5.9 million pairs/second
        \item Fidelity $>$90\%
    \end{itemize}

    \item \textbf{Quantum Teleportation Payload:} Bell-state measurement capability
\end{enumerate}

% ============================================================================
% SECTION 4.2: MICIUS EXPERIMENTS
% ============================================================================

\section{Micius Experimental Results}
\label{sec:micius_experiments}

\subsection{Satellite-to-Ground QKD (2017)}

Liao et al. \cite{liao2017micius_qkd} demonstrated the first satellite-to-ground QKD, published in Nature:

\textbf{Experimental Configuration:}
\begin{itemize}
    \item Ground station: Xinglong, near Beijing
    \item Distance range: 507 km to 1034 km
    \item Measurement duration: 273 seconds per pass
\end{itemize}

\textbf{Key Results:}
\begin{table}[h]
\centering
\caption{Micius QKD Performance Results}
\label{tab:micius_qkd_results}
\begin{tabular}{lcc}
\toprule
\textbf{Metric} & \textbf{530 km} & \textbf{1034 km} \\
\midrule
Sifted key rate & 40.2 kbps & 1.2 kbps \\
Secure key rate & 12.0 kbps & 0.4 kbps \\
QBER & 1.1\% & 3.2\% \\
Channel loss & 21.5 dB & 41.5 dB \\
\bottomrule
\end{tabular}
\end{table}

\textbf{Significance:} Demonstrated that satellite QKD can exceed ground-based fiber performance for distances $>$400 km.

\subsection{Entanglement Distribution (2017)}

Yin et al. \cite{yin2017entanglement_1200km} achieved record-breaking entanglement distribution, published in Science:

\textbf{Configuration:}
\begin{itemize}
    \item Two ground stations: Delingha and Lijiang
    \item Separation: 1203 km
    \item Simultaneous detection of entangled pairs
\end{itemize}

\textbf{Results:}
\begin{itemize}
    \item Bell inequality violation: $S = 2.37 \pm 0.09 > 2$
    \item Detection rate: 1.1 pairs/second
    \item Fidelity: 87.4\%
\end{itemize}

\textbf{Significance:} Demonstrated entanglement preservation over unprecedented distances, opening possibility for device-independent QKD.

\subsection{Intercontinental QKD (2018)}

Liao et al. \cite{liao2018intercontinental} demonstrated China-Austria QKD, published in Physical Review Letters:

\textbf{Configuration:}
\begin{itemize}
    \item Ground stations: China (multiple) and Austria (Graz)
    \item Total distance: 7600 km
    \item Satellite as trusted relay node
\end{itemize}

\textbf{Demonstration:}
\begin{itemize}
    \item 75-minute encrypted videoconference between Beijing and Vienna
    \item 128-bit AES keys exchanged via QKD
    \item First intercontinental quantum-secured communication
\end{itemize}

\subsection{Full Quantum Teleportation (2021)}

Pan's team demonstrated full quantum state teleportation over 1200 km ground distance using satellite-distributed entanglement.

% ============================================================================
% SECTION 4.3: INTEGRATED NETWORK
% ============================================================================

\section{Integrated Space-Ground Network (2021)}
\label{sec:integrated_network}

\subsection{Chen et al. (2021) - Nature}

Chen et al. \cite{chen2021integrated_network} published ``An integrated space-to-ground quantum communication network over 4,600 kilometres'' in Nature, representing the most comprehensive quantum network demonstration.

\textbf{Network Architecture:}
\begin{itemize}
    \item \textbf{Fiber backbone:} 2000 km Beijing-Shanghai link
    \item \textbf{Satellite links:} 2600 km via Micius
    \item \textbf{Metropolitan networks:} 4 QMANs (Shanghai, Hefei, Jinan, Beijing)
    \item \textbf{Trusted nodes:} 32 fiber relay stations
    \item \textbf{Users:} 150+ across government, banking, grid
\end{itemize}

\textbf{Performance:}
\begin{table}[h]
\centering
\caption{Integrated Network Performance}
\label{tab:integrated_network}
\begin{tabular}{lcc}
\toprule
\textbf{Link Type} & \textbf{Key Rate} & \textbf{Improvement} \\
\midrule
Satellite-ground & 47.8 kbps & 40$\times$ vs. previous \\
Fiber backbone & Variable & Continuous operation \\
Metropolitan & High rate & Local applications \\
\bottomrule
\end{tabular}
\end{table}

\textbf{Applications Demonstrated:}
\begin{itemize}
    \item Banking transactions (ICBC, Bank of China)
    \item Power grid communications (State Grid)
    \item Government e-services
    \item Encrypted voice/video conferencing
\end{itemize}

% ============================================================================
% SECTION 4.4: OTHER SATELLITE MISSIONS
% ============================================================================

\section{Other Satellite QKD Missions}
\label{sec:other_missions}

\subsection{Jinan-1 Micro-Satellite (2022)}

China launched the Jinan-1 micro-nano satellite in July 2022, demonstrating cost-effective QKD:
\begin{itemize}
    \item Smaller form factor than Micius
    \item Compact ground stations
    \item 2025: Achieved QKD with South Africa (12,900 km)
\end{itemize}

\subsection{Planned Missions (2025-2027)}

\begin{table}[h]
\centering
\caption{Upcoming Satellite QKD Missions}
\label{tab:future_missions}
\begin{tabular}{llll}
\toprule
\textbf{Mission} & \textbf{Country} & \textbf{Launch} & \textbf{Objective} \\
\midrule
Eagle-1 & ESA & 2025-2026 & Operational QKD service \\
QUBE-II & Germany & 2025 & CubeSat BB84 demonstration \\
QEYSSat & Canada & 2025 & LEO constellation study \\
Next-gen Micius & China & 2025 & LEO constellation (2-3 sats) \\
MEO satellite & China & 2027 & Extended coverage \\
\bottomrule
\end{tabular}
\end{table}

\subsection{Japanese NICT Experiments}

Japan's NICT has conducted ground-to-LEO experiments using the SOCRATES satellite, demonstrating:
\begin{itemize}
    \item Uplink QKD feasibility
    \item Pointing acquisition and tracking
    \item Photon transmission through atmosphere
\end{itemize}

% ============================================================================
% SECTION 4.5: COMPARISON WITH NGUYEN ET AL.
% ============================================================================

\section{Comparison with Nguyen et al. (2021)}
\label{sec:comparison_nguyen}

\subsection{System Parameter Comparison}

\begin{table}[h]
\centering
\caption{Nguyen et al. vs. Micius Systems}
\label{tab:nguyen_vs_micius}
\small
\begin{tabular}{lcc}
\toprule
\textbf{Parameter} & \textbf{Nguyen (2021)} & \textbf{Micius (2017)} \\
\midrule
\multicolumn{3}{l}{\textit{System Configuration}} \\
Satellite altitude & 600 km & 500 km \\
Protocol & QPSK-based & Decoy-state BB84 \\
Wavelength & 1550 nm & 850 nm \\
Modulation & Phase (QPSK) & Polarization \\
Detection & Heterodyne (APD) & Single-photon (Si-APD) \\
\midrule
\multicolumn{3}{l}{\textit{Error Handling}} \\
Method & ARQ retransmission & LDPC FEC \\
Complexity & Low & Medium \\
Adaptability & Channel-adaptive & Fixed rate \\
\midrule
\multicolumn{3}{l}{\textit{Validation}} \\
Status & Simulation & Experimental \\
\bottomrule
\end{tabular}
\end{table}

\subsection{Complementary Contributions}

While Micius demonstrated experimental feasibility, Nguyen et al. addresses:

\begin{enumerate}
    \item \textbf{Alternative Detection:} Coherent detection vs. single-photon
    \item \textbf{Reliability Mechanism:} ARQ vs. FEC approach
    \item \textbf{Analytical Framework:} 3-D Markov model for link-layer analysis
    \item \textbf{Wavelength Choice:} Telecom-band (1550 nm) for compatibility
\end{enumerate}

% ============================================================================
% SECTION 4.6: LESSONS LEARNED
% ============================================================================

\section{Lessons from Satellite Experiments}
\label{sec:lessons}

\subsection{Technical Insights}

\begin{enumerate}
    \item \textbf{Downlink Preferred:} Satellite-to-ground links experience less turbulence impact than uplinks

    \item \textbf{Night Operation:} Current systems operate primarily at night to avoid solar background noise

    \item \textbf{Pointing Critical:} Sub-microradian pointing accuracy essential for stable links

    \item \textbf{LEO Optimal:} Low Earth orbit provides best balance of loss and pass duration

    \item \textbf{Trusted Nodes:} Practical networks currently require trusted relay satellites
\end{enumerate}

\subsection{Relevance to Nguyen et al.}

The experimental lessons inform Nguyen et al.'s design choices:

\begin{itemize}
    \item \textbf{LEO Configuration:} 600 km altitude consistent with optimal range
    \item \textbf{Downlink Scenario:} Satellite-to-ground transmission
    \item \textbf{Atmospheric Modeling:} Gamma-Gamma distribution validated by experiments
    \item \textbf{Error Handling:} Retransmission addresses practical channel variability
\end{itemize}

% ============================================================================
% SECTION 4.7: CHAPTER SUMMARY
% ============================================================================

\section{Chapter Summary}
\label{sec:ch4_summary}

This chapter reviewed satellite QKD experimental achievements:

\begin{enumerate}
    \item \textbf{Micius Satellite:} First dedicated quantum satellite, demonstrating QKD, entanglement distribution, and teleportation

    \item \textbf{Performance Benchmarks:} 40.2 kbps sifted key rate, 1.1\% QBER, 1200 km entanglement

    \item \textbf{Integrated Network:} 4600 km space-ground network with 150+ users

    \item \textbf{Future Missions:} Eagle-1, QEYSSat, and expanded Chinese constellation

    \item \textbf{Context for Nguyen et al.:} Experimental validation supports theoretical assumptions; novel contributions address detection and reliability challenges
\end{enumerate}

\textbf{Gap Identification:} Experimental work has focused on DV-QKD with single-photon detection. Nguyen et al.'s coherent detection and retransmission approach offers an alternative paradigm requiring future experimental validation.
