% ============================================================================
% CHAPTER 2: DETAILED PAPER ANALYSIS
% ============================================================================

\chapter{Detailed Paper Analysis}
\label{chap:paper_analysis}

\begin{quote}
\textit{This chapter provides a comprehensive technical analysis of Nguyen et al. (2021), examining the system architecture, key innovations, mathematical framework, and main contributions to satellite-based QKD reliability improvement.}
\end{quote}

% ============================================================================
% SECTION 2.1: SYSTEM ARCHITECTURE
% ============================================================================

\section{System Architecture Overview}
\label{sec:system_architecture}

\subsection{Two-Layer Design}

The proposed system operates across two layers:

\begin{enumerate}
    \item \textbf{Physical Layer:} Responsible for quantum key transmission over FSO channel
    \begin{itemize}
        \item QPSK modulator with Mach-Zehnder modulators (MZMs)
        \item FSO channel with combined loss mechanisms
        \item Dual-threshold/heterodyne detection (DT/HD) receiver
    \end{itemize}

    \item \textbf{Link Layer:} Manages key retransmission for reliability
    \begin{itemize}
        \item Buffer management at Alice (satellite)
        \item ACK/NACK feedback via classical RF channel
        \item Retransmission protocol with maximum $M$ attempts
    \end{itemize}
\end{enumerate}

\subsection{Link Configuration}

\begin{table}[h]
\centering
\caption{System Configuration}
\label{tab:system_config}
\begin{tabular}{lll}
\toprule
\textbf{Component} & \textbf{Location} & \textbf{Function} \\
\midrule
Alice (Transmitter) & LEO Satellite (600 km) & Key generation, QPSK modulation \\
Bob (Receiver) & Ground Station (5 m) & DT/HD detection, key recovery \\
Forward Channel & FSO (Downlink) & Quantum key transmission \\
Feedback Channel & RF (Classical) & ACK/NACK signaling \\
\bottomrule
\end{tabular}
\end{table}

% ============================================================================
% SECTION 2.2: KEY INNOVATIONS
% ============================================================================

\section{Key Innovations}
\label{sec:innovations}

\subsection{Innovation 1: QPSK-Based QKD Protocol}

The paper adapts Quadrature Phase-Shift Keying (QPSK) for quantum key distribution:

\textbf{Advantages over alternatives:}
\begin{itemize}
    \item Compared to SIM/BPSK: No RF subcarrier required, simpler implementation
    \item Compared to polarization encoding: Compatible with coherent detection
    \item Direct mapping of BB84 four-state structure to four QPSK phase states
\end{itemize}

\textbf{Phase State Mapping:}
\begin{equation}
\phi_A \in \left\{ \frac{\pi}{4}, \frac{3\pi}{4}, \frac{5\pi}{4}, -\frac{\pi}{4} \right\}
\label{eq:phase_states}
\end{equation}

\subsection{Innovation 2: Dual-Threshold/Heterodyne Detection}

The DT/HD receiver combines two techniques:

\textbf{Heterodyne Detection:}
\begin{itemize}
    \item Signal mixed with strong local oscillator
    \item Improved receiver sensitivity compared to direct detection
    \item Both quadratures accessible (though only one used for key)
\end{itemize}

\textbf{Dual-Threshold Decision:}
\begin{equation}
\text{Decision} = \begin{cases}
0 & \text{if } i \geq d_0 \\
1 & \text{if } i \leq d_1 \\
X & \text{otherwise (erasure)}
\end{cases}
\label{eq:dual_threshold}
\end{equation}

\textbf{Quantified Improvement:} 20 dB reduction in required transmitted power compared to SIM/BPSK-DT.

\subsection{Innovation 3: Key Retransmission Scheme}

The ARQ-based retransmission scheme operates as follows:

\begin{enumerate}
    \item Alice generates key sequence, stores in buffer
    \item Transmits via FSO channel to Bob
    \item Bob checks received sequence for errors
    \item If successful: Send ACK, Alice removes from buffer
    \item If failed: Send NACK, Alice retransmits (up to $M$ times)
    \item After $M$ failures: Discard sequence, count as key loss
\end{enumerate}

\textbf{Advantage over FEC:}
\begin{itemize}
    \item No computational overhead for encoding/decoding
    \item Adapts naturally to channel variations
    \item Simple implementation
\end{itemize}

\subsection{Innovation 4: 3-D Markov Chain Model}

Novel analytical framework with three-dimensional state space:

\begin{equation}
\text{State: } (n, s, m) \text{ where}
\begin{cases}
n \in [0, C] & \text{Buffer queue length} \\
s \in \{B, G\} & \text{Channel state (Bad/Good)} \\
m \in [1, M] & \text{Retransmission attempt number}
\end{cases}
\label{eq:markov_state}
\end{equation}

This model enables analytical calculation of Key Loss Rate (KLR).

% ============================================================================
% SECTION 2.3: MAIN CONTRIBUTIONS
% ============================================================================

\section{Main Contributions Summary}
\label{sec:contributions}

\begin{table}[h]
\centering
\caption{Paper Contributions and Impact}
\label{tab:contributions}
\begin{tabular}{p{3.5cm}p{4cm}p{4cm}}
\toprule
\textbf{Contribution} & \textbf{Type} & \textbf{Impact} \\
\midrule
QPSK-based QKD with DT/HD & System Design & 20 dB power improvement \\
Key retransmission scheme & Protocol Innovation & $>$1000$\times$ KLR reduction \\
3-D Markov chain model & Analytical Framework & Enables KLR prediction \\
Comprehensive channel model & Mathematical Derivation & Realistic performance analysis \\
Parameter optimization & Numerical Results & Practical design guidelines \\
\bottomrule
\end{tabular}
\end{table}

% ============================================================================
% SECTION 2.4: KEY RESULTS
% ============================================================================

\section{Key Results Summary}
\label{sec:key_results}

\subsection{Physical Layer Results}

\textbf{QBER Performance:}
\begin{itemize}
    \item Achieves QBER $< 10^{-3}$ with proper DT coefficient selection
    \item Optimal $\varsigma$ range: 0.7--2.4 (weak turbulence), 1.4--2.8 (strong turbulence)
    \item $P_{sift} \geq 10^{-2}$ maintained for sufficient key rate
\end{itemize}

\textbf{Power Comparison:}
\begin{table}[h]
\centering
\caption{Required Transmitted Power for QBER $\leq 10^{-3}$}
\label{tab:power_comparison}
\begin{tabular}{lc}
\toprule
\textbf{Scheme} & \textbf{Required $P_T$} \\
\midrule
SIM/BPSK-DT & 45 dBm \\
QPSK-DT/DD & 35 dBm \\
\textbf{QPSK-DT/HD (Proposed)} & \textbf{25 dBm} \\
\bottomrule
\end{tabular}
\end{table}

\subsection{Link Layer Results}

\textbf{KLR Improvement with Retransmissions:}
\begin{table}[h]
\centering
\caption{Key Loss Rate vs. Number of Retransmissions}
\label{tab:klr_results}
\begin{tabular}{ccc}
\toprule
\textbf{Retransmissions ($M$)} & \textbf{KLR} & \textbf{Improvement} \\
\midrule
0 (Conventional) & $3 \times 10^{-2}$ & Baseline \\
1 & $10^{-3}$ -- $10^{-2}$ & 10$\times$ \\
2 & $10^{-4}$ -- $10^{-3}$ & 100$\times$ \\
4 & $<10^{-4}$ & $>$1000$\times$ \\
\bottomrule
\end{tabular}
\end{table}

\textbf{Key Finding:} Diminishing returns beyond $M=4$; only 0.5 dB additional power gain from $M=4$ to $M=7$.

\subsection{Security Analysis}

\textbf{Unauthorized Receiver Attack (URA):}
\begin{itemize}
    \item Eve's QBER increases with distance from Bob
    \item Minimum secure distance: $D_{E-B} > 30$ m (both weak and strong turbulence)
    \item Security maintained when Eve's QBER $> 10^{-2}$
\end{itemize}

% ============================================================================
% SECTION 2.5: SYSTEM PARAMETERS
% ============================================================================

\section{System Parameters}
\label{sec:parameters}

\begin{table}[h]
\centering
\caption{Complete System Parameters}
\label{tab:parameters}
\small
\begin{tabular}{llll}
\toprule
\textbf{Category} & \textbf{Parameter} & \textbf{Symbol} & \textbf{Value} \\
\midrule
\multirow{4}{*}{Physical Constants} & Electron charge & $q$ & $1.6 \times 10^{-19}$ C \\
& Boltzmann constant & $k_B$ & $1.38 \times 10^{-23}$ W/K/Hz \\
& Planck's constant & $\tilde{h}$ & $6.63 \times 10^{-34}$ J$\cdot$s \\
\midrule
\multirow{7}{*}{Receiver} & Bit rate & $R_b$ & 10 Gbps \\
& Load resistor & $R_L$ & 50 $\Omega$ \\
& Excess noise factor & $x$ & 0.8 \\
& Avalanche multiplication & $\bar{g}$ & 10 \\
& Responsivity & $\Re$ & 0.8 \\
& Temperature & $T$ & 298 K \\
& Dark current & $I_d$ & 3 nA \\
\midrule
\multirow{9}{*}{Channel} & Wavelength & $\lambda$ & 1550 nm \\
& Satellite altitude & $H_S$ & 600 km \\
& Ground station height & $H_G$ & 5 m \\
& Atmospheric altitude & $H_\beta$ & 20 km \\
& Zenith angle & $\zeta$ & 50$^\circ$ \\
& Wind speed & $w$ & 21 m/s \\
& Beam width & $\omega_D$ & 50 m \\
& Detection aperture & $a$ & 0.31 m \\
& Attenuation coefficient & $\gamma$ & 0.43 dB/km \\
\midrule
\multirow{2}{*}{Telescope} & Tx gain & $G_T$ & 120 dB \\
& Rx gain & $G_R$ & 121 dB \\
\midrule
\multirow{2}{*}{Link Layer} & Flow throughput & $H$ & 185 seq/s \\
& Bit sequence length & $l_{bs}$ & $3 \times 10^6$ bits \\
\bottomrule
\end{tabular}
\end{table}
