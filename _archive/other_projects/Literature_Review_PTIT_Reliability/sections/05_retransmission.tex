% ============================================================================
% CHAPTER 5: RETRANSMISSION SCHEME
% ============================================================================

\chapter{Key Retransmission Scheme}
\label{chap:retransmission}

\begin{quote}
\textit{This chapter analyzes the key retransmission scheme proposed by Nguyen et al. (2021), including the protocol design, quantum channel-state model, 3-D Markov chain analysis, and Key Loss Rate derivation.}
\end{quote}

% ============================================================================
% SECTION 5.1: MOTIVATION
% ============================================================================

\section{Motivation for Retransmission}
\label{sec:retransmission_motivation}

\subsection{Limitations of FEC}

Traditional QKD systems use Forward Error Correction (FEC) for information reconciliation:

\textbf{Disadvantages of FEC:}
\begin{enumerate}
    \item \textbf{Computational overhead:} Complex encoding/decoding algorithms
    \item \textbf{Memory requirements:} Large buffers for block processing
    \item \textbf{Fixed redundancy:} Cannot adapt to varying channel conditions
    \item \textbf{Limited error capability:} Can only correct up to code's design limit
    \item \textbf{Efficiency reduction:} Redundancy reduces effective key rate
\end{enumerate}

\subsection{Advantages of ARQ Retransmission}

The proposed key retransmission scheme offers:
\begin{itemize}
    \item \textbf{No computational overhead:} Simple ACK/NACK mechanism
    \item \textbf{Adaptive:} Naturally responds to channel variations
    \item \textbf{Simple implementation:} Standard ARQ techniques
    \item \textbf{Flexible:} Adjustable maximum retransmission count $M$
\end{itemize}

% ============================================================================
% SECTION 5.2: PROTOCOL DESIGN
% ============================================================================

\section{Protocol Design}
\label{sec:protocol_design}

\subsection{System Components}

\textbf{At Satellite (Alice):}
\begin{itemize}
    \item Key generator: Produces random bit sequence $d(t)$
    \item Buffer: Stores sequences pending transmission/acknowledgment
    \item Transmitter: QPSK modulator
    \item Retransmission controller: Manages retransmission count
\end{itemize}

\textbf{At Ground Station (Bob):}
\begin{itemize}
    \item Receiver: DT/HD detector
    \item Error checker: Verifies received sequence integrity
    \item Feedback transmitter: Sends ACK/NACK via RF channel
\end{itemize}

\subsection{Protocol Operation}

\begin{enumerate}
    \item \textbf{Transmission:}
    \begin{itemize}
        \item Alice generates bit sequence of length $l_{bs}$
        \item Sequence stored in buffer, forwarded to transmitter
        \item Transmitted over FSO channel
    \end{itemize}

    \item \textbf{Reception and Verification:}
    \begin{itemize}
        \item Bob receives and decodes sequence
        \item Checks for errors in sifted key
        \item Determines success/failure
    \end{itemize}

    \item \textbf{Feedback:}
    \begin{itemize}
        \item \textbf{Success (no errors):} Bob sends ACK
        \item \textbf{Failure (errors detected):} Bob sends NACK
    \end{itemize}

    \item \textbf{Alice's Response:}
    \begin{itemize}
        \item \textbf{On ACK:} Remove sequence from buffer
        \item \textbf{On NACK:} Increment retransmission counter; if $m < M$, retransmit
        \item \textbf{On $m = M$:} Discard sequence (counts as key loss)
    \end{itemize}
\end{enumerate}

\subsection{Key Loss Scenarios}

Sequences can be lost due to:
\begin{enumerate}
    \item \textbf{Maximum retransmissions exceeded:} $M$ failed attempts
    \item \textbf{Buffer overflow:} New sequence arrives when buffer full
\end{enumerate}

% ============================================================================
% SECTION 5.3: CHANNEL STATE MODEL
% ============================================================================

\section{Quantum Channel-State Model}
\label{sec:channel_state}

\subsection{Two-State Markov Model}

The quantum channel alternates between two states:
\begin{itemize}
    \item \textbf{Good (G):} All sifted keys transmitted error-free
    \item \textbf{Bad (B):} Transmission fails due to errors
\end{itemize}

\subsection{Quantum Key Error Rate (QKER)}

The probability that a bit sequence transmission fails:
\begin{equation}
\text{QKER} = 1 - (1 - \text{QBER})^{l_{bs} \cdot P_{sift}}
\label{eq:qker}
\end{equation}

where:
\begin{itemize}
    \item $l_{bs} = 3 \times 10^6$ bits: Length of bit sequence
    \item $P_{sift}$: Probability of sifting (from QPSK protocol)
    \item QBER: Quantum bit error rate
\end{itemize}

\subsection{State Transition Probabilities}

\begin{align}
p_{BB} &= \text{QKER} \cdot \left(1 - \frac{\tau_{bs}}{\tau_0}\right) \label{eq:p_bb} \\
p_{GG} &= (1 - \text{QKER}) \cdot \left(1 - \frac{\tau_{bs}}{\tau_0}\right) \label{eq:p_gg} \\
p_{BG} &= 1 - p_{BB} \label{eq:p_bg} \\
p_{GB} &= 1 - p_{GG} \label{eq:p_gb}
\end{align}

where:
\begin{itemize}
    \item $\tau_{bs} = l_{bs}/R_b$: Time slot duration
    \item $\tau_0 = \sqrt{\lambda D_\beta / w}$: Turbulence coherent time
\end{itemize}

% ============================================================================
% SECTION 5.4: 3-D MARKOV CHAIN
% ============================================================================

\section{Queue-Associated 3-D Markov Chain}
\label{sec:markov_chain}

\subsection{State Space Definition}

The system state at each time slot is defined by three dimensions:
\begin{equation}
\text{State: } (n, s, m)
\label{eq:state_definition}
\end{equation}

where:
\begin{itemize}
    \item $n \in [0, C]$: Number of bit sequences in buffer
    \item $s \in \{B, G\}$: Channel state (Bad or Good)
    \item $m \in [0, M]$: Retransmission attempt number for current sequence
\end{itemize}

\textbf{Note:} States with $n = 0$ and $m > 0$ are impossible (no sequence to retransmit).

\subsection{State Transitions}

The transitions depend on:
\begin{enumerate}
    \item \textbf{Arrival process:} Bernoulli with probability $H\tau_{bs}$
    \item \textbf{Channel state:} Markov with transition probabilities $p_{ss'}$
    \item \textbf{Transmission outcome:} Success (G) removes sequence, Failure (B) triggers retransmission
\end{enumerate}

\subsection{Transition Probability Matrix}

Key transition cases:

\textbf{Case 1: Empty buffer, bad channel}
\begin{itemize}
    \item $(0, B, 0) \rightarrow (1, B, 0)$: $H\tau_{bs} \cdot p_{BB}$ (arrival, stay bad)
    \item $(0, B, 0) \rightarrow (0, G, 0)$: $(1-H\tau_{bs}) \cdot p_{BG}$ (no arrival, become good)
\end{itemize}

\textbf{Case 2: Non-empty buffer, bad channel, not at max retransmissions}
\begin{itemize}
    \item $(n, B, m) \rightarrow (n+1, B, m+1)$: $H\tau_{bs} \cdot p_{BB}$ (arrival, fail again)
    \item $(n, B, m) \rightarrow (n, G, m+1)$: $(1-H\tau_{bs}) \cdot p_{BG}$ (no arrival, become good)
\end{itemize}

\textbf{Case 3: Good channel state}
\begin{itemize}
    \item Transmission succeeds, sequence removed
    \item Buffer decrements, retransmission counter resets
\end{itemize}

\subsection{Balance Equations}

The steady-state probabilities $\pi(n, s, m)$ satisfy:
\begin{equation}
\begin{cases}
\boldsymbol{\Pi}^T \mathbf{P} = \boldsymbol{\Pi}^T \\
\sum_{n=0}^{C} \sum_{s \in \{B,G\}} \sum_{m=0}^{M} \pi(n, s, m) = 1
\end{cases}
\label{eq:balance}
\end{equation}

where $\mathbf{P}$ is the transition probability matrix of size $(C+1) \times 2 \times (M+1)$.

Solution via standard numerical methods (Jacobi iteration or Gauss elimination).

% ============================================================================
% SECTION 5.5: KEY LOSS RATE
% ============================================================================

\section{Key Loss Rate Derivation}
\label{sec:klr_derivation}

\subsection{KLR Expression}

The Key Loss Rate accounts for losses from:
\begin{enumerate}
    \item Maximum retransmissions exceeded (M failed attempts)
    \item Buffer overflow
\end{enumerate}

\begin{equation}
\text{KLR} = \sum_{s \in \{B,G\}} \sum_{m=0}^{M} \pi(C, s, m) + \sum_{n=0}^{C-1} \pi(n, B, M)
\label{eq:klr}
\end{equation}

\textbf{First term:} Buffer full states (overflow loss)
\textbf{Second term:} States at maximum retransmission with bad channel (discard loss)

\subsection{Physical Interpretation}

\begin{itemize}
    \item $\pi(C, s, m)$: Probability of buffer being full --- any new arrival causes overflow
    \item $\pi(n, B, M)$: Probability of being at max retransmission in bad channel --- sequence discarded
\end{itemize}

\subsection{Optimization Insight}

From the analysis:
\begin{itemize}
    \item Increasing $M$ reduces KLR (more chances to succeed)
    \item But: Diminishing returns beyond $M \approx 4$
    \item Trade-off: Larger $M$ means longer latency for failed sequences
\end{itemize}

% ============================================================================
% SECTION 5.6: COMPARISON WITH FEC
% ============================================================================

\section{Comparison with FEC Approaches}
\label{sec:fec_comparison}

\begin{table}[h]
\centering
\caption{Retransmission vs. FEC for QKD Error Handling}
\label{tab:retx_vs_fec}
\begin{tabular}{lcc}
\toprule
\textbf{Aspect} & \textbf{Retransmission (Proposed)} & \textbf{FEC (Traditional)} \\
\midrule
Computational load & Low & High \\
Memory requirement & Buffer only & Large code blocks \\
Adaptability & Automatic & Fixed redundancy \\
Latency (good channel) & Low & Fixed \\
Latency (bad channel) & Variable & Fixed \\
Implementation & Simple & Complex \\
Efficiency & Variable & Fixed (rate loss) \\
\bottomrule
\end{tabular}
\end{table}

\textbf{Conclusion:} Retransmission is particularly suitable for satellite QKD where:
\begin{itemize}
    \item Channel conditions vary during satellite pass
    \item Computational resources on satellite may be limited
    \item Simplicity improves reliability
\end{itemize}
